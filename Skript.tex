\documentclass[11pt, twoside, a4paper]{article}

% Setup
\usepackage[margin=2.4cm, top=3.5cm]{geometry}
\usepackage[utf8]{inputenc}
\usepackage[ngerman]{babel}

% Package imports
\usepackage{amsfonts}
\usepackage{amsmath}
\usepackage{amssymb}
\usepackage{amsthm}
\usepackage{mathtools}
\usepackage{setspace}
\usepackage{float}
\usepackage{enumitem}
\usepackage{hyperref}
\usepackage[pagestyles]{titlesec}
\usepackage{fancyhdr}
\usepackage{colonequals}
\usepackage{caption}
\usepackage{tikz}
\usepackage{marginnote}
\usepackage{etoolbox}
\usepackage{mdframed}
\usepackage{aligned-overset}

% Font-Encoding
\usepackage[T1]{fontenc}
\usepackage{lmodern}

% Theorems
\newtheorem{blockelement}{Blockelement}[subsection]
\newtheoremstyle{plain}{}{}{}{}{\bfseries}{.}{ }{}
\theoremstyle{plain}
\newtheorem{bemerkung}[blockelement]{Bemerkung}
\newtheorem{definition}[blockelement]{Definition}
\newtheorem{lemma}[blockelement]{Lemma}
\newtheorem{satz}[blockelement]{Satz}
\newtheorem{notation}[blockelement]{Notation}
\newtheorem{korollar}[blockelement]{Korollar}
\newtheorem{uebung}[blockelement]{Übung}
\newtheorem{beispiel}[blockelement]{Beispiel}
\newtheorem{folgerung}[blockelement]{Folgerung}
\newtheorem{axiom}[blockelement]{Axiom}
\newtheorem{beobachtung}[blockelement]{Beobachtung}
\newtheorem{konzept}[blockelement]{Konzept}
\newtheorem{visualisierung}[blockelement]{Visualisierung}
\newtheorem{anwendung}[blockelement]{Anwendung}
\newtheorem{skizze}[blockelement]{Skizze}

% Marginnotes left
\makeatletter
\patchcmd{\@mn@@@marginnote}{\begingroup}{\begingroup\@twosidefalse}{}{\fail}
\reversemarginpar
\makeatother

% Long equations
\allowdisplaybreaks

% \left \right
\newcommand{\set}[1]{\left\{#1\right\}}
\newcommand{\pair}[1]{\left(#1\right)}
\newcommand{\of}[1]{\mathopen{}\mathclose{}\bgroup\left(#1\aftergroup\egroup\right)}
\newcommand{\abs}[1]{\left\lvert#1\right\rvert}
\newcommand{\norm}[1]{\left\lVert#1\right\rVert}
\newcommand{\linterv}[1]{\left[#1\right)}
\newcommand{\rinterv}[1]{\left(#1\right]}
\newcommand{\interv}[1]{\left[#1\right]}
\newcommand{\sprod}[1]{\left<#1\right>}

% Shorten commands
\newcommand{\equivalent}[0]{\Leftrightarrow{}}
\newcommand{\impl}[0]{\Rightarrow{}}
\newcommand{\fromto}{\rightarrow{}}
\newcommand{\definedas}[0]{\coloneqq}
\newcommand{\definedasbackwards}[0]{\eqqcolon}
\newcommand{\definedasequiv}[0]{\ratio\Leftrightarrow{}}
\newcommand{\exclude}[0]{\setminus}
\renewcommand{\emptyset}{\varnothing}
\newcommand{\sbset}{\subseteq}

\newcommand{\ntoinf}[0]{n\fromto\infty}
\newcommand{\toinf}{\fromto\infty}
\newcommand{\fa}{\;\forall\,}
\newcommand{\ex}{\;\exists\,}
\newcommand{\conj}[1]{\overline{#1}}

\newcommand{\annot}[3][]{\overset{\text{#3}}#1{#2}}
\newcommand{\biglim}[1]{{\displaystyle \lim_{#1}}}
\newcommand{\nn}[0]{\\[2\baselineskip]}
\newcommand{\anf}[1]{\glqq{}#1\grqq}
\newcommand{\OBDA}{o.B.d.A. }
\newcommand{\theoremescape}{\leavevmode}
\newcommand{\aligntoright}[2]{\hfill#1\hspace{#2\textwidth}~}
\newcommand{\horizontalline}[0]{\par\noindent\rule{0.05\textwidth}{0.1pt}\\}
\newcommand{\rgbcolor}[3]{rgb,255:red,#1;green,#2;blue,#3}
\newcommand{\fixedspace}[2]{\makebox[#1][l]{#2}}

\let\Re\relax
\let\Im\relax

% MathOperators
\DeclareMathOperator{\grad}{Grad}
\DeclareMathOperator{\bild}{Bild}
\DeclareMathOperator{\Re}{Re}
\DeclareMathOperator{\Im}{Im}

% Mengenbezeichner
\newcommand{\R}{\mathbb{R}}
\newcommand{\N}{\mathbb{N}}
\newcommand{\C}{\mathbb{C}}
\newcommand{\Z}{\mathbb{Z}}
\newcommand{\Q}{\mathbb{Q}}
\newcommand{\K}{\mathbb{K}}

\newcommand\imaginarysubsection[1]{
    \refstepcounter{subsection}
    \subsectionmark{#1}
}

% Unfassbar hässlich, aber effektiv für temporäre schnelle Lösungen
\def\:={\coloneqq}
\def\->{\fromto}
\def\=>{\impl}
\def\<={\leq}
\def\>={\geq}
\def\!={\neq}

% Envs
\newenvironment{induktionsanfang}{
    \rule{0pt}{3ex}\noindent
    \begin{minipage}[t]{0.11\textwidth}
    {I-Anfang}
    \end{minipage}
    \hfill
    \begin{minipage}[t]{0.89\textwidth}
    }
    {
    \end{minipage}
}
\newenvironment{induktionsvoraussetzung}{
    \rule{0pt}{3ex}\noindent
    \begin{minipage}[t]{0.11\textwidth}
    {I-Vor.}
    \end{minipage}
    \hfill
    \begin{minipage}[t]{0.89\textwidth}
    }
    {
    \end{minipage}
}
\newenvironment{induktionsschritt}{
    \rule{0pt}{3ex}\noindent
    \begin{minipage}[t]{0.11\textwidth}
    {I-Schritt}
    \end{minipage}
    \hfill
    \begin{minipage}[t]{0.89\textwidth}
    }
    {
    \end{minipage}
}

% Section style
\titleformat*{\section}{\LARGE\bfseries}
\titleformat*{\subsection}{\large\bfseries}

% Page styles
\newpagestyle{pagenumberonly}{
    \sethead{}{}{}
    \setfoot[][][\thepage]{\thepage}{}{}
}
\newpagestyle{headfootdefault}{
    \sethead[][][\thesubsection~\textit{\subsectiontitle}]{\thesection~\textit{\sectiontitle}}{}{}
    \setfoot[][][\thepage]{\thepage}{}{}
}
\pagestyle{headfootdefault}

\begin{document}
    \title{\vspace{3cm} Skript zur Vorlesung\\Analysis I\\bei Prof. Dr. Dirk Hundertmark}
    \author{Karlsruher Institut für Technologie}
    \date{Wintersemester 2023/24}
    \maketitle
    \begin{center}
        Dieses Skript ist inoffiziell. Es besteht kein\\ Anspruch auf Vollständigkeit oder Korrektheit.
    \end{center}
    \thispagestyle{empty}
    \newpage

    \tableofcontents
    ~\\
    Alle mit [*] markierten Kapitel sind noch nicht korrektur gelesen und bedürfen eventuell noch Änderungen.
    \newpage


    \section{Aussagenlogik}
    \input{Kapitel/Aussagenlogik}


    \section{Mengen}
    \input{Kapitel/Mengen}


    \section{Die Axiome der reellen Zahlen}
    \thispagestyle{pagenumberonly}

Es gibt eine Menge $\R$, genannt reelle Zahlen, die 3 Gruppen von Axiomen erfüllt:
\begin{enumerate}
    \item Algebraische Axiome
    \item Anordnungsaxiome
    \item Das Vollständigkeitsaxiom
\end{enumerate}

\subsection{Algebraische Axiome}
In $\R$ gibt es 2 Operationen:
\begin{enumerate}
    \item Addition \anf{+}
    \item Multiplikation \anf{$\cdot$}
\end{enumerate}

\begin{folgerung}
    $a,b\in\R\impl a+b\in \R$ und $a \cdot b\in\R$
\end{folgerung}

\begin{definition}[Eigenschaften eines Körpers]
    \theoremescape
    \begin{enumerate}[label=(I.\arabic*)]
        \item $(a+b) + c = a + (b+c)$\quad \textit{Assoziativität der Addition}
        \item $a + b = b + a$\quad \textit{Kommutativität der Addition}
        \item Es gibt genau eine Zahl genannt Null, geschrieben $0$, mit $\forall a \in\R\colon a+0 = a$ \quad \textit{Existenz eines neutralen Elements der Addition}
        \item $\forall a\in\R~\exists! b\in \R\colon a + b = 0$, geschrieben $b=-a$\quad \textit{Existenz eines inversen Elements der Addition}
        \item $\pair{a\cdot b}\cdot c = a\cdot\pair{b\cdot c}$\quad \textit{Assoziativität der Multiplikation}
        \item $a\cdot b = b \cdot a$\quad \textit{Kommutativität der Multiplikation}
        \item $\forall a \in\R, a \neq 0$ gibt es ein eindeutiges $b\neq 0$ mit $a\cdot b = 1$. Wir schreiben $b = a^{-1} = \frac{1}{a}$\quad \textit{Existenz eines inversen Elements der Multiplikation}
        \item Es gibt genau eine Zahl Eins, geschrieben 1, die von 0 verschieden ist, mit $\forall a\in\R\colon a\cdot 1 = a$\quad \textit{Existenz eines neutralen Elements der Multiplikation}
        \item $a\cdot\pair{b+c} = a \cdot b + a \cdot c$\quad \textit{Distributivität}
    \end{enumerate}

    \noindent Jede Menge $\K$, welche (I.1) bis (I.9) erfüllt, heißt \textbf{Körper}.
\end{definition}

\begin{bemerkung}
    Dass die Eindeutigkeit von 0 und 1 durch die Axiome gefordert wird, ist nicht unbedingt erforderlich.\footnote{Seien $0, 0'$ neutrale Elemente bezüglich der Addition. $\impl 0 = 0+0' = 0'+0=0'$}
\end{bemerkung}

\begin{bemerkung}
    Das inverse Elemente bezüglich Addition und Multiplikation ist eindeutig.
    \begin{proof}
        Annahme: $a+b=0$ und $a+b'=0$
        \begin{align*}
            &\impl b + 0 = b + \pair{a+b'} = b' + \pair{a+b} = b' + 0\\
            &\impl b = b'\qedhere
        \end{align*}
    \end{proof}
\end{bemerkung}

\begin{notation}
    \begin{align*}
        a-b &\definedas a+(-b)\tag{Differenz}\\[10pt]
        \frac{a}{b} &\definedas a \cdot b^{-1}\tag{Quotient}
    \end{align*}
\end{notation}

\begin{satz}[Abgeleitete Regeln]
    Es gilt\\
    (I.10)
    \begin{align}
        -\pair{-a} &= a\\[8pt]
        \pair{-a} + \pair{-b} &= -\pair{a+b}\\[8pt]
        \pair{a^{-1}}^{-1} &= a\\[8pt]
        a^{-1}\cdot b^{-1} &= \pair{a\cdot b}^{-1}\\[8pt]
        a\cdot 0 &= 0\\[8pt]
        a\cdot\pair{-b} &= -\pair{a\cdot b}\\[8pt]
        (-a)\cdot(-b) &= a\cdot b\\[8pt]
        a\cdot\pair{b-c} &= a\cdot b - a\cdot c
    \end{align}
    \noindent (I.11) Ist $a\cdot b = 0$ so ist mindestens eine der Zahlen $a$ oder $b$ gleich Null.

    \begin{proof}[Beweis zu (I.10.5)]
        Zu zeigen: $a \cdot 0 = 0$
        \begin{align*}
            a\cdot 0 + a \cdot 0 &= a \cdot\pair{0+0}\\
            &= a \cdot 0\\
            \impl \pair{a\cdot 0 + a \cdot 0} + \pair{-a \cdot 0} &= a \cdot 0 + \pair{-a\cdot 0}\\
            \impl a \cdot 0 + \pair{a\cdot 0 + \pair{-a\cdot 0}} &= 0\\
            \impl a\cdot 0 +0 &= 0\\
            \impl a\cdot 0 &= 0\qedhere
        \end{align*}
    \end{proof}
    \begin{proof}[Beweis zu (I.11)]
        Sei $a\cdot b = 0$.\\
        Ist $a\neq 0 \impl b = 1\cdot b = a^{-1}\cdot a \cdot b = a^{-1}\cdot (a\cdot b) = a^{-1} = 0 = 0$\\
        Ist $b\neq 0,$ so gilt analog, dass $a = 0$.
    \end{proof}
    \begin{uebung}
        Beweisen Sie die verbleibenden Regeln aus (I.10).
    \end{uebung}
\end{satz}

\begin{satz}[Regeln des Bruchrechnens]
    \label{satz:bruchrechnen}
    ~\\(I.12) Es gilt:
    \setcounter{equation}{0}
    \begin{alignat}{4}
        \frac{a}{b} + \frac{c}{d} &= \frac{ad+cb}{bd} &\text{ für } b,d&\neq 0\\[10pt]
        \frac{a}{b} \cdot \frac{c}{d} &= \frac{ac}{bd} &\text{ für } b,d&\neq 0\\[10pt]
        \frac{\frac{a}{b}}{\frac{c}{d}} &= \frac{ad}{bc} &\text{ für } b,c,d&\neq 0
    \end{alignat}
    \begin{uebung}
        Beweisen Sie Satz~\ref{satz:bruchrechnen}.
    \end{uebung}
\end{satz}

\subsection{Die Anordnungsaxiome}

Allgemein gilt: $a,b\in\R \impl a = b \lor a \neq b$.\\
Ist $a\neq b$, besteht eine Anordnung \anf{$<$}, die verlangt, dass genau eine der Relationen $a<b$ oder $b<a$ gilt.
Das heißt $\forall a,b\in\R$ gilt genau eine der Aussagen $a<b$, $b<a$, $a=b$.\\
Diese Anordnung genügt folgenden Axiomen:
\begin{axiom}[Anordnungsaxiome]
    \theoremescape
    \begin{enumerate}[label=(II.\arabic*)]
        \item $a<b \land b < c \impl a < c$\quad \textit{Transitivität}
        \item $a<b,~c \in\R \impl a + c < b + c$
        \item $a<b,~c > 0 \impl ac < bc$
    \end{enumerate}
\end{axiom}

\begin{notation}
    \theoremescape
    \begin{enumerate}[label=-]
        \item $a < b$: a ist (echt) kleiner als b
        \item $b > a$: b ist größer als a
        \item $a\leq b$: $a=b$ oder $a < b$
        \item $a\in\R$ ist positiv, wenn $a>0$; negativ, wenn $a <0$; nicht-negativ, wenn $a\geq 0$; nicht-positiv, wenn $a\leq 0$
    \end{enumerate}
\end{notation}

\begin{beispiel}
    $a<b\equivalent b - a > 0$
    \begin{proof}
        \begin{align*}
            a &<b\\
            \impl 0 = a + \pair{-a} &< b + \pair{-a} = b - a\\[10pt]
            b-a &>0\\
            \impl a &< a + \pair{b-a} = b\qedhere
        \end{align*}
    \end{proof}
\end{beispiel}

%%%%%%%%%%%%%%%%%%%%%%%%
% 7. November 2023
%%%%%%%%%%%%%%%%%%%%%%%%

\begin{satz}[Aus den Anordnungsaxiomen abgeleitete Regeln]
    \marginnote{[7. Nov]}
    \theoremescape
    \begin{enumerate}[label=(II.\arabic*)]
        \setcounter{enumi}{3}
        \item $a<b\equivalent b-a > 0$
        \item $a<0\equivalent -a > 0$ und $a>0\equivalent -a < 0$
        \item $a<b\equivalent -b < -a$
        \item $a<b \land c < d \equivalent a+c<b+d$
        \item $ab > 0 \equivalent \pair{a>0 \land b > 0}\lor\pair{a < 0 \land b < 0}$ und $ab < 0 \equivalent \pair{a>0 \land b < 0}\lor\pair{a < 0 \land b > 0}$
        \item $a\neq 0 \impl a^2 > 0$\quad(Insbesondere $1>0$)
        \item $a<b \land c<0 \impl ac > bc$
        \item $a>0 \equivalent \frac{1}{a}>0$
        \item $a^2 < b^2 \land a > 0 \land b > 0 \impl a < b$
    \end{enumerate}
    \newpage
    \begin{proof}
        \theoremescape
        \begin{enumerate}[label=(II.\arabic*)]
            \setcounter{enumi}{3}
            \item Sei $a<b \impl 0=a+(-a) \annot{<}{(II.2)} b + (-a) = b-a$.\\
            Ist $b-a>0 \annot{\impl}{(II.2)} a<a+(b-a)=b$
            \item Setze $b\definedas 0$ in (II.4) $\impl b-a=-a>0$.\\
            2ter Teil: Ersetze $a$ durch $-a$ in (II.5). ($a>0 \impl -a < 0 \equivalent -(-a)>0 \equivalent a >0$)
            \item (II.6) folgt aus (II.5), da $a<b\equivalent b-a>0 \equivalent (-a)-(-b) > 0\equivalent -b < -a$
            \item Sei $a<b \land c < d \annot{\impl}{(II.2)} a + c < b + c \land b + c < b + d \annot{\impl}{(II.1)} a+c < b + d$
            \item $a,b>0 \annot{\impl}{(II.3)} ab > 0\cdot b = 0$ und $a,b<0 \annot{\impl}{(II.5)} -a,-b>0 \impl (-a)(-b) > 0 \impl ab > 0$.\\
            Umkehrung: Sei $ab>0 \impl a\neq 0 \land b \neq 0$. Wäre $a>0 \land b < 0 \annot{\impl}{(II.5)} -b>0$. Wie gerade gezeigt folgt $a(-b) > 0 \impl -ab > 0 \annot{\impl}{(II.5)} ab < 0$ (Widerspruch zur Annahme).\\
            Genauso zeigt man, dass die Annahme $a<0 \land b > 0$ falsch ist.\\
            (Zweite Behauptung lässt sich analog zeigen).
            \item $a\neq 0 \equivalent a > 0 \lor a < 0 \annot{\impl}{(II.8)} a^2 = a \cdot a > 0$. Ferner ist $1\neq 0 \impl 1=1\cdot 1 > 0$
            \item Sei $c<0 \impl -c > 0$ und aus $a<b$ folgt $(-c)\cdot a < (-c)\cdot b \impl -c\cdot a < -c\cdot b \impl c\cdot b < c\cdot a$
            \item $a\cdot a^{-1} = 1 > 0$ (falls $a\neq 0$) $\annot{\impl}{(II.8)} a^{-1} > 0$ sofern $a>0$ ist und aus $a^{-1}>0$ folgt $a>0$
            \item Sei $a^{2}<b^2, a>0, b>0$. Angenommen $a<b$ ist falsch, d.h. $a\geq b \impl a^{2} \geq a \cdot a \geq a\cdot b\geq b\cdot b = b^{2}\impl a^{2}\geq b^{2}$ (Widerspruch)
        \end{enumerate}
    \end{proof}
\end{satz}

\newpage

\subsection{Das Vollständigkeitsaxiom}

\begin{axiom}[Vollständigkeitsaxiom]
    \label{axiom:vollstaendigkeitsaxiom}
    Jede nicht-leere Teilmenge $M\subseteq \R$, welche nach oben beschränkt ist, besitzt eine kleinste obere Schranke, genannt das Supremum von $M$.
\end{axiom}

\begin{notation}[Supremum]
    Das Supremum einer Menge $M$ schreiben wir als $\sup M$.
\end{notation}

\begin{definition}[Beschränktheit von Mengen]
    Sei $M\subseteq \R, M\neq \emptyset$.
    \begin{enumerate}[label=(\roman*)]
        \item $M$ heißt \textbf{nach oben beschränkt}, falls ein $k\in\R$ existiert mit $\forall x\in M\colon x\leq k$.
        Jede solche Zahl $k$ heißt obere Schranke von $M$.
        \item $M$ heißt \textbf{nach unten beschränkt}, falls ein $k\in\R$ existiert mit $\forall x\in M\colon x\geq k$.
        Jede solche Zahl $k$ heißt untere Schranke von $M$.
        \item $M$ heißt \textbf{beschränkt}, falls ein $k\geq 0$ existiert mit $-k\leq x \leq k\quad \forall x\in M$
    \end{enumerate}
\end{definition}

\begin{definition}[Kleinste obere und größte untere Schranke]
    Eine Zahl $k\in\R$ heißt kleinste obere (größte untere) Schranke, falls
    \begin{enumerate}
        \item es eine obere (untere) Schranke ist und
        \item es keine kleinere obere (größere untere) Schranke für $M$ gibt
    \end{enumerate}
\end{definition}

\begin{folgerung}
    \theoremescape
    Allgemein gilt
    \begin{align*}
        x \leq k \equivalent -k \leq -x
    \end{align*}
    das heißt für eine Menge $M\neq\emptyset$ gilt
    \begin{center}
        $k$ ist eine obere Schranke für $M$\\ $\equivalent -k$ ist eine untere Schranke für $-M\definedas\set{-x|~x\in M}$
    \end{center}
    und
    \begin{center}
        $k$ ist kleinste obere Schranke für M\\ $\equivalent -k$ ist die größte untere Schranke für $-M$
    \end{center}
    Das heißt das Vollständigkeitsaxiom ist äquivalent zum \textit{Vollständigkeitsaxiom}$^{-1}$ (Jede nicht-leere Teilmenge $M\subseteq \R$, welche nach unten beschränkt ist, besitzt eine größte untere Schranke, genannt das Infimum von $M$. Wir schreiben $\inf M$).
\end{folgerung}

\begin{beispiel}
    \theoremescape
    \begin{align*}
        M &\definedas \interv{0,1} = \set{x|~0\leq x \leq 1}\\
        \sup M &= 1 \qquad \inf M = 0\nn
        A &\definedas \pair{0,1} = \set{x|~0< x < 1}\\
        \sup A &= 1 \qquad \inf A = 0\\
    \end{align*}
\end{beispiel}

\begin{notation}
    Sei $M\subseteq \R, M\neq \emptyset$\\
    Wir schreiben $\sup M < \infty$, falls $M$ nach oben beschränkt ist, andernfalls setzen wir
    \begin{align*}
        \sup M \definedas \infty
    \end{align*}
    Falls $M$ nach unten beschränkt ist, schreiben wir $\inf M > -\infty$, andernfalls setzen wir
    \begin{align*}
        \inf M \definedas -\infty
    \end{align*}
\end{notation}

\begin{satz}[Eigenschaften des Supremums]
    \label{satz:sup}
    Sei $M\subseteq \R$
    \begin{enumerate}[label=(\roman*)]
        \item Ist $\sup M < \infty$, so folgt $\forall \varepsilon > 0~\exists x\in M$ mit $\sup\pair{M} -\varepsilon < x$
        \item Ist $\sup M = \infty$, so gilt $\forall k\geq 0~\exists x\in M$ mit $x> k$
    \end{enumerate}
    \begin{proof}
        \theoremescape
        \begin{enumerate}[label=(\roman*)]
            \item Wir setzen $a\definedas \sup M$. Sei $a<\infty$. Wäre (i) falsch, so folgt $\exists \varepsilon>0~\forall x\in M\colon a-\varepsilon > x$.\\Das heißt $a-\varepsilon$ ist eine obere Schranke für $M$. Aber $a-\varepsilon < a$ (Widerspruch)
            \item Ist $a=\infty$, so hat $M$ keine obere Schranke. Nach Def. folgt für jedes $k\in\R$ existiert ein $x\in M\colon x > k$\qedhere
        \end{enumerate}
    \end{proof}
\end{satz}
\begin{satz}[Eigenschaften des Infimums]
    \label{satz:inf}
    Sei $M\subseteq \R$
    \begin{enumerate}[label=(\roman*)]
        \item Ist $\inf M > -\infty$, so folgt $\forall \varepsilon > 0~\exists x\in M$ mit $x < \inf\pair{M}+ \varepsilon$
        \item Ist $\inf M = -\infty$, so gilt $\forall k\geq 0~\exists x\in M$ mit $x< -k$
    \end{enumerate}
    \begin{proof}
        Wende Satz~\ref{satz:sup} auf $-M\definedas\set{-x|~x\in M}$ an und beachte $\sup\pair{-M} = \inf\pair{M}$
    \end{proof}
\end{satz}
\begin{definition}[Maximum und Minimum]
    Es sei $M\subseteq \R, M\neq\emptyset$. $m\in M$ heißt größtes Element von $M$ (Maximum), geschrieben $\max M$, falls
    \begin{align*}
        x\leq m\quad\forall x\in M
    \end{align*}
    Entsprechend: $m\in M$ heißt kleinstes Element von $M$ (Minimum), geschrieben $\min M$, falls
    \begin{align*}
        x\geq m\quad\forall x\in M
    \end{align*}
\end{definition}

\begin{beispiel}
    Sei $M$ beschränkt, $M\neq\emptyset$
    \begin{align*}
        M&\definedas\set{x|~0\leq x< 1}\\
        \sup M &= 1\\
        \inf M &= \min M = 0\\
        M&\text{ hat kein Maximum}
    \end{align*}
\end{beispiel}

\newpage


    \section{Die natürlichen Zahlen $\N$ und vollständige Induktion}
    \input{Kapitel/Natuerliche_Zahlen}


    \section{Summe, Produkt, Wurzeln}
    %%%%%%%%%%%%%%%%%%%%%%%%
% 14. November 2023
%%%%%%%%%%%%%%%%%%%%%%%%

\thispagestyle{pagenumberonly}

\subsection{Summenzeichen, Produktzeichen}
\begin{definition}
    \marginnote{[14. Nov]}
    Seien $m\leq n$, $m,n\in\N_{0}$. Für jedes $k\in\N_0$, $m\leq k\leq n$, sei $a_k\in\R$.\\
    Dann setzt man:
    \begin{align*}
        \sum_{k=m}^{n}a_k &= a_m+a_{m+1}+a_{m+2}+\dots+a_{n}
        \intertext{und}
        \prod_{k=m}^{n} &= a_m\cdot a_{m+1}\cdot a_{m+2}\cdot \ldots\cdot a_{n}
    \end{align*}
    Für $n\in\N_0$, $n<m$ setzt man $\prod_{k=m}^{n}a_k = 1$.
\end{definition}
\begin{definition}[Fakultät]
    Sei $n\in\N$, dann gilt:
    \begin{align*}
        n! &= 1 \cdot 2\cdot 3 \cdot \ldots \cdot n
        \intertext{und wir definieren}
        0! &= 1
    \end{align*}
    Alternativ lässt sich rekursiv definieren:
    \begin{align*}
        0! &= 1\\
        n! &= (n-1)! \cdot n
    \end{align*}
\end{definition}

\begin{satz} % Satz 3
    Die Anzahl aller möglichen Anordnungen einer $n$-elementigen Menge $\set{A_1, \dots, A_n}$ ist gleich $n!$.\\
    Wenn wir beispielsweise die Menge $\set{1,2,3}$ betrachten. Mögliche Anordnungen: $\set{1,2,3}$, $\set{1,3,2}$, $\set{2,1,3}$, $\set{2,3,1}$, $\set{3,1,2}$, $\set{3,2,1}$. Somit gibt es 6 Möglichkeiten, was $3!$ entspricht.
    \begin{proof}[Induktionsbeweis]
        ~\\
        \begin{induktionsanfang}
            $n=1$, es gibt eine Anordnung $\set{A_1}$ und es gilt $1! = 1$
        \end{induktionsanfang}
        \\
        \begin{induktionsschritt}
            Die Gesamtzahl aller Anordnungen von $\set{A_1, \dots, A_{n+1}}$ ist gleich
            \begin{align*}
                &(n+1)\cdot [\text{Gesamtzahl von Anordnungen von }\set{A_1, \dots, A_n}]\\
                \annot{=}{I-Ann} &(n+1) \cdot n! = (n+1)!\qedhere
            \end{align*}
        \end{induktionsschritt}
    \end{proof}
\end{satz}

\subsection{Binomischer Lehrsatz}
\begin{definition}[Binomialkoeffizient]
    Für $n,k\in\N_0$ setzt man:
    \begin{align*}
        \binom{n}{k} &\definedas \frac{n\cdot(n-1)\cdot\dots\cdot(n-k+1)}{k!} = \frac{n!}{k!\cdot(n-k)!} \tag{$n$ über $k$}
    \end{align*}
\end{definition}
\begin{bemerkung}[Spezielle Binomialkoeffizienten]
    $\binom{n}{0} = 1, \binom{n}{n} = 1, \binom{n}{k} = 0$ für $k>n$
\end{bemerkung}

\begin{satz}
    \label{satz:teilmengen-anzahl}
    Die Anzahl der $k$-elementigen Teilmengen einer $n$-elementigen Menge $\set{A_1, \dots, A_n}$ ist gleich $\binom{n}{k}$.
\end{satz}

\begin{lemma}
    \label{lemma:binom-add}
    $\forall k,n\in\N$ gilt $\binom{n}{k} = \binom{n-1}{k-1} + \binom{n-1}{k}$.
    \begin{proof}
        \begin{align*}
            \binom{n-1}{k} + \binom{n-1}{k-1} &= \frac{(n-1)!}{k!\cdot(n-1-k)!} + \frac{(n-1)!}{(k-1)!\cdot(n-1-k+1)!}\\
            &= \frac{(n-1)!\cdot(n-k)}{k!\cdot(n-k)!} + \frac{(n-1)!\cdot k}{k!\cdot(n-k)!}\\
            &= \frac{(n-1)!\cdot\interv{n-k+k}}{k!\cdot(n-k)!}\\
            &= \frac{n!}{k!\cdot(n-k)!} \annot{=}{Def.} \binom{n}{k}\qedhere
        \end{align*}
    \end{proof}
\end{lemma}
\begin{proof}[Beweis von Satz~\ref{satz:teilmengen-anzahl} (Induktion nach $n$)]
    ~\\
    \begin{induktionsanfang}
        $n=1$, $\set{A_1}$. Wenn $k=0$, dann gibt es eine Möglichkeit und es gilt $\binom{1}{0} = 1$. Wenn $k=1$, gibt es auch eine Möglichkeit und es gilt $\binom{1}{1} = 1$.\\
    \end{induktionsanfang}
    \\
    \begin{induktionsschritt}
        $n\rightarrow n+1$\\
        Die Behauptung sei für $M_n=\set{A_1, \dots, A_n}$ schon bewiesen. Wir betrachten $M_n+1 = \set{A_1, \dots, A_{n+1}}$. Für $k=0$ und $k=n+1$ ist die Behauptung offensichtlich.\\
        Für $1\leq k \leq n$ gehört jede $k$-elementige Teilmenge von $M_{n+1}$ zu genau einer der folgenden Klassen:
        \begin{enumerate}
            \item $T_0$ besteht aus $k$-elementigen Teilmengen, die $A_{n+1}$ nicht enthalten.
            \item $T_1$ besteht aus denjenigen Teilmengen, die $A_{n+1}$ enthalten.
        \end{enumerate}
        \noindent In $T_0$ gibt es nach Induktionsannahme $\binom{n}{k}$ Elemente.\\
        In $T_1$ gibt es $\binom{n}{k-1}$ Elemente\footnotemark.\\
        Insgesamt:
        \begin{equation*}
            \binom{n}{k}+\binom{n}{k-1}\annot{=}{\ref{lemma:binom-add}}\binom{n+1}{k}\qedhere
        \end{equation*}
    \end{induktionsschritt}
    \footnotetext{Wir wissen, dass $A_{n+1}$ bereits ein Element der Teilmenge ist. Damit müssen wir noch $k-1$ aus $n$ Elemente auswählen. Die Formel dafür folgt aus der Induktionsannahme}
\end{proof}

\begin{satz}[Binomischer Lehrsatz]
    \label{satz:binom-lehrsatz}
    Sei $x,y\in\R$ und $n\in\N$. Dann gilt:
    \begin{align*}
    (x+y)
        ^{n} &= \sum_{k=0}^{n} \binom{n}{k}\cdot x^{n-k}\cdot y^k
    \end{align*}
\end{satz}
\begin{beispiel}[Folgerung der binomischen Formel aus dem binomischen Lehrsatz]
    Es sei $n=2$. Es gilt $\binom{2}{0}=1$, $\binom{2}{1}=2$, $\binom{2}{2}=1$. Daraus folgt:
    \begin{align*}
    (x+y)
        ^{2} &= x^2+2xy+y^2
    \end{align*}
\end{beispiel}
\begin{proof}[Beweis von Satz~\ref{satz:binom-lehrsatz}]
    ~\\IA: $n=0$
    \begin{align*}
    (x+y)
        ^0&=1\\
        \sum_{k=0}^0\binom{0}{k}\cdot x^{k}\cdot y^{0-k} &= \binom{0}{k}\cdot 1\cdot 1 = 1
    \end{align*}
    Induktionsschritt: $n\rightarrow n+1$
    \begin{align*}
    (x+y)
        ^{n+1} &= (x+y)^n\cdot (x+y) = (x+y)^n \cdot x + (x+y)^n\cdot y\\
        (x+y)^n\cdot x &\annot{=}{I-An} \sum_{k=0}^{n}\binom{n}{k}\cdot x^{n-k}\cdot y^k\cdot x\\
        &=1\cdot x^{n+1} + \sum_{k=1}^{n} \binom{n}{k}\cdot x^{n+1-k}\cdot y^{k}\\
        (x+y)^n\cdot y &= \sum_{k=0}^{n}\binom{n}{k}\cdot x^{n-k}\cdot y^{k+1}\tag{$l\definedas k+1$}\\
        &= \sum_{l=1}^{n+1} \binom{n}{l-1}\cdot x^{n+1-l}\cdot y^{l}\\
        &= \sum_{k=1}^{n+1} \binom{n}{k-1}\cdot x^{n+1-k} \cdot y^{k}\\[10pt]
        \impl (x+y)^{n+1} &= x^{n+1} + \sum_{k=1}^{n} \interv{\binom{n}{k}+\binom{n}{k-1}} \cdot x^{n+1-k}\cdot y^k + y^{n+1}\\
        &\annot{=}{\ref{lemma:binom-add}} \sum_{k=0}^{n+1} \binom{n+1}{k}\cdot x^{n+1-k}\cdot y^{k}\qedhere
    \end{align*}
\end{proof}

\begin{bemerkung}
    Sei $x>0$, dann gilt $(1+x)^n = 1+\underbrace{\binom{n}{1}x}_{n\cdot x} + \underbrace{\sum \dots}_{>0} > 1 + n\cdot x$
\end{bemerkung}

\vfill

\subsection{Bernoullische Ungleichung}

\begin{satz}
    \label{satz:bernoulli-ungleichung}
    Es sei $n\in\N$ und $a\in\R$, $a > -1$. Dann gilt
    \begin{align*}
    (1+a)
        ^n \geq 1+na
    \end{align*}
    \begin{proof}
        Wir verwenden vollständige Induktion:\\
        \begin{induktionsanfang}
            $n=1 \impl 1+a = 1+a$
        \end{induktionsanfang}
        \\
        \begin{induktionsschritt}
            $n\rightarrow n+1$
            \begin{align*}
            (1+a)
                ^{n+1} &= (1+a)^n\cdot (1+a) \annot{\geq}{I-Ann} (1+na)\cdot(1+a) \\
                &= 1+na+a+na^2\geq 1+(n+1)\cdot a\qedhere
            \end{align*}
        \end{induktionsschritt}
    \end{proof}
\end{satz}

\vfill

%%%%%%%%%%%%%%%%%%%%%%%%
% 16. November 2023
%%%%%%%%%%%%%%%%%%%%%%%%

\newpage

\subsection{Wurzeln}

\begin{satz}[Existenz und Eindeutigkeit der Quadratwurzel]
    \marginnote{[16. Nov]}
    Für jedes $c\in\R$, $c>0$, gibt es genau ein $x>0$, so dass $x^2 = c$ ist.
    \begin{proof}
        \textit{Eindeutigkeit}\\
        $x_1>0$, $x_2>0$: $\pair{x_1}^2 = \pair{x_2}^2 = c \impl 0 = (\pair{x_1}^2-\pair{x_2}^2) = (x_1-x_2) \cdot \underbrace{(x_1+x_2)}_{>0} \impl x_1 = x_2$\\
        \textit{Existenz.} Wir definieren $M\definedas\set{z\in\R|~z\geq 0, z^2 \leq c}$. Dann gilt $0\in M \impl M\neq \emptyset$\\[10pt]
        $M$ ist beschränkt, weil $(1+c)^2=1+2c+c^2 > c$, \quad$z\in M \impl z < 1 + c$\\
        Somit $\exists \sup M$ und wir definieren $x\definedas \sup M$. Zu zeigen: $x^2 = c$\\[10pt]
        Wir nehmen an, dass $x^2<c$ und setzen $\varepsilon \definedas \min\set{1, \frac{c-x^2}{2x+1}} \impl 0 < \varepsilon \leq 1 \impl \varepsilon^2 < \varepsilon$\\
        $(x+\varepsilon)^2 = x^2 + 2\varepsilon x + \varepsilon^2 < x^2+\varepsilon\pair{2x+1}\leq x^2+c-x^2=c$\\
        $\impl x+\varepsilon\in M$ (Widerspruch) $\impl x^2 \geq c$\\[10pt]
        Wir nehmen an, dass $x^2 > c$, $\varepsilon \definedas\min\set{\frac{x^2-c}{2x}, \frac{x}{2}}$, $\varepsilon > 0$, $x-\varepsilon \geq x-\frac{x}{2}>0$\\
        $\pair{x-\varepsilon}^2 = x^2 - 2 x\varepsilon + \varepsilon^2 > x^2-2x\varepsilon \geq x^2-x^2+c\impl (x-\varepsilon)^2 > c \impl x\neq \sup M$ (Widerspruch)\\[10pt]
        $\impl x^2 = c$
    \end{proof}
\end{satz}

\begin{bemerkung}
    $x=\sqrt {c}$, $x=c^{\frac{1}{2}}$, $x$ ist die Quadratwurzel von $c$
\end{bemerkung}

\begin{satz}[Existenz und Eindeutigkeit der Wurzel]
    Für $n\in\N$ und für jedes $c\in\R$, $c\geq 0$ gibt es genau ein $x \geq 0$, $x\in\R$, so dass $x^n = c$.
    \begin{proof}
        \textit{Eindeutigkeit}\\
        $x_1>0$, $x_2>0$, $\pair{x_1}^n=\pair{x_2}^n = c$, $0=\pair{\pair{x_1}^n-\pair{x_2}^n} = \pair{x_1 - x_2}\cdot\pair{\sum_{k=0}^{n-1} \pair{x_1}^{n-k+1}\cdot \pair{x_2}^k}$\\
        $\impl x_1 = x_2$
    \end{proof}
    \noindent Die Existenz ist Aufgabe auf dem Übungsblatt.
\end{satz}

\begin{definition}[Spezielle Potenzen]
    $m,n\in\N$\quad $x^\frac{m}{n}\definedas \pair{x^\frac{1}{m}}^n$, $x^0 = 1$, $0^m = 0$ mit $m\neq 0$, $0^0 = 1$
\end{definition}

\subsection{Absolutbetrag}

\begin{definition}[Betrag]
    $\abs{a} \definedas \left\{ \begin{array}{lr}
                                    a  & a>0 \\
                                    0  & a=0 \\
                                    -a & a<0
    \end{array}\right.$
\end{definition}

\begin{satz}[Eigenschaften des Betrags]
    \theoremescape
    \begin{enumerate}[label=(\roman*)]
        \item $\abs{a} \geq 0$, $\abs{a} = 0 \equivalent a = 0$
        \item $\abs{\lambda\cdot a} = \abs{\lambda}\cdot\abs{a}$\quad $\forall \lambda, a \in \R$
        \item $\abs{a+b}\leq \abs{a}+\abs{b}$ (Dreiecksungleichung) %%% 5.14
    \end{enumerate}
    \begin{proof}[Beweis von (iii)]
        \begin{align*}
            \abs{a+b}^2 &= \pair{a+b}^2 = a^2 + 2ab + b^2\\
            &\leq \abs{a}^2 + 2\abs{a}\abs{b} + \abs{b}^2 = \pair{\abs{a}+\abs{b}}^2\\
            \impl \abs{a+b}&\leq\abs{a}+\abs{b}\qedhere
        \end{align*}
    \end{proof}
\end{satz}

\begin{definition}[Geometrische Betrachtung des Betrags]
    Man nennt $\abs{a-b}$ den Abstand zweier Punkte $a,b\in\R$ auf der Zahlengerade.
\end{definition}

\begin{satz}[Eigenschaften von Differenzen im Betrag]
    \label{satz:diff-abs}
    \theoremescape
    \begin{enumerate}[label=(\roman*)]
        \item $\abs{a-b}\geq 0$, $\abs{a-b} = 0\equivalent a = b$
        \item $\abs{a-b} = \abs{b-a}$
        \item $\abs{a-b}\leq \abs{a-c} + \abs{b-c}$\quad $\forall c\in\R$ %%% 5.15
    \end{enumerate}
    \begin{proof}[Beweis von (iii)]
        \begin{align*}
            \abs{a-b} &= \abs{a-c+c-b} \leq \abs{a-c}+\abs{c-b} = \abs{a-c}+\abs{b-c}\qedhere
        \end{align*}
    \end{proof}
\end{satz}

\begin{satz} %%% 5.16
    $\forall a,b\in\R$ gilt $\abs{\abs{a}-\abs{b}}\leq\abs{a-b}$
    \begin{proof}
        \begin{align*}
            \abs{a} &= \abs{a-b+b} \leq \abs{a-b} + \abs{b}\\
            &\impl
            \begin{array}{l}
                \abs{a}-\abs{b} \leq \abs{a-b} \\
                \abs{b}-\abs{a} \leq \abs{b-a} = \abs{a-b}
            \end{array}
            \\[10pt]
            &\impl \abs{a-b}\geq\abs{\abs{a}-\abs{b}}\qedhere
        \end{align*}
    \end{proof}
\end{satz}

\begin{folgerung}
    \theoremescape
    \begin{enumerate}[label=(\roman*)]
        \item $\abs{a-b}\geq \abs{a}-\abs{b}$
        \item $\abs{a+b} = \abs{a-\pair{-b}} \geq \abs{a} - \abs{b}$
    \end{enumerate}
\end{folgerung}

\begin{bemerkung}
    Durch Induktion leitet man her, dass
    \begin{align*}
        \abs{\sum_{i=1}^{n} a_i} &\leq \sum_{i=1}^{n} \abs{a_i}\quad a_i \in\R
    \end{align*}
\end{bemerkung}

\newpage


    \section{Folgen und Grenzwerte}
    \input{Kapitel/Folgen_Grenzwerte}


    \section{Dichtheit von $\Q$ in $\R$}
    \input{Kapitel/Dichtheit}


    \section{Reihen (und Konvergenz von Reihen)}
    \input{Kapitel/Reihen}


    \section{$\R^d$, Konvergenz im $\R^d$, die komplexen Zahlen $\C$ und der Raum $\C^d$}
    \input{Kapitel/Raum_Rd}


    \section{Polynome}
    \imaginarysubsection{Reelle Polynome}
\thispagestyle{pagenumberonly}

\begin{definition}[Reelles Polynom]
    Es sei $a_0,a_1, \dots, a_n\in\R$,~$a_n \neq 0,~\in\R$. Dann ist
    \begin{align*}
        P(x) &\definedas a_0 + a_1 \cdot x + a_2 \cdot x^2 + \dots + a_n \cdot x^n
    \end{align*}
    ein reelles Polynom vom Grad $n$ mit $\grad(P) = n$.
\end{definition}

\begin{definition}[Nullpolynom]
    $P(x) = 0$ ist das Nullpolynom. Ein Polynom $P$ ist nicht-trivial, wenn es nicht das Nullpolynom ist.
\end{definition}

\begin{bemerkung}[Analytische Polynome]
    Ähnlich zu reellen Polynomen lassen sich auch analytische Polynome definieren. Es sei $a_0, a_1, \dots, a_n \in\C,~a_0\neq 0,~z\in\C$. Dann ist
    \begin{align*}
        P(z) &\definedas a_0 + a_1\cdot z + a_2\cdot z^2 + \dots + a_n\cdot z^n
    \end{align*}
    eine analytisches Polynom mit $\grad(P) = n$.
\end{bemerkung}

\begin{definition}[Grad von speziellen Polynomen]
    Wir definieren den Grad von konstanten Polynomen der Form $P(x) = a_0$ als 0 und den Grad des Nullpolynoms als $-1$.
\end{definition}

\begin{satz}[Eigenschaften von Polynomen]
    \theoremescape
    \label{satz:eigenschaften-polynome}
    \begin{enumerate}[label=(\roman*)]
        \item Sei $P$ ein Polynom mit $\grad(P) = n$ und $\lambda\in\R,~\lambda\neq 0$. Dann ist $\grad(\lambda P) = n$.
        \item Seien $P, Q$ nicht-triviale Polynome mit $\grad(P) = n$ und $\grad(Q)=m$. Dann gilt $PQ$ ist ein Polynom mit $\grad(PQ)=n+m$.
        \begin{proof}
            \begin{align*}
                P(x) &= \sum_{j=0}^{n} a_j x^j\\
                Q(x) &= \sum_{i=0}^{n} b_i x^i\\
                P(x) \cdot Q(x) &= \pair{\sum_{j=0}^{n} a_j x^j}\cdot \pair{\sum_{i=0}^{n} b_i x^i}\\
                &= \sum_{j=0}^{n} \sum_{l=0}^{m} a_{j} b_l\cdot x^{j+l}
                \intertext{Wir setzen $k= j+l\in\set{0,1,\dots,n+m}$}
                &= \sum_{k=0}^{n+m} \pair{\sum_{\substack{j=0\\ 0\leq k-j\leq m}}^{n} a_j b_{k-j}} x^k\\
                &= a_n b_m\cdot x^{n+m} + \textit{Terme niedrigerer Ordnung}\qedhere
            \end{align*}
        \end{proof}
        \newpage
        \item Entwicklung in einem anderen Punkt. Wir schreiben $x = \eta + \zeta$
        \begin{align*}
            P(x) &= \sum_{j=0}^{n} a_j x^j = \sum_{j=0}^{n} a_j \cdot\pair{\eta + \zeta}^j\\
            \annot[{&}]{=}{\ref{satz:binom-lehrsatz}} \sum_{j=0}^{n} a_j \cdot\pair{\sum_{l=0}^{j} \binom{j}{l}\cdot\eta^{l}\cdot\zeta^{j-l}}
            \intertext{Es gilt $\binom{j}{l}=0$, falls $l>j$. Also können wir auch schreiben}
            &= \sum_{j=0}^{n} a_j \cdot\pair{\sum_{l\geq 0} \binom{j}{l}\cdot\eta^{l}\cdot\zeta^{j-l}}\\
            &= \sum_{l\geq 0} \pair{\sum_{j=0}^{n} a_j\cdot \binom{j}{l}\cdot\zeta^{j-l}}\cdot\eta^l\\
            &= \sum_{l\geq 0} \underbrace{\pair{\sum_{j=0}^{n} \binom{j}{l}\cdot a_j\cdot\zeta^{j-l}}}_{\definedasbackwards b_l}\cdot\pair{x-\zeta}^l\\
            &= \sum_{l=0}^{n} b_l \cdot\pair{x-\zeta}^l
            \intertext{Wir betrachten $l=0$}
            b_0 &= \sum_{j=0}^{n} \binom{j}{0}\cdot a_j\cdot\zeta^{j} = \sum_{j=0}^{n} a_j \zeta^j = P(\zeta)\\
            \impl P(x) &= P(\zeta) + \sum_{l=1}^{n} b_l \cdot\pair{x-\zeta}^l\\
            &= P(\zeta) + (x-\zeta) \cdot \underbrace{\sum_{l=0}^{n-1} b_{l+1} \cdot \pair{x-\zeta}^l}_{\definedasbackwards Q(x-\zeta)}\\
            &= P(\zeta) + (x-\zeta)\cdot Q(x-\zeta)
        \end{align*}
        Dabei gilt außerdem $\grad(Q) = n-1$. Diesen Zusammenhang werden wir später in Satz~\ref{satz:nullstellen-polynome} verwenden, um eine Aussage über die Anzahl an Nullstellen eines Polynoms zu treffen.
    \end{enumerate}
\end{satz}

%%%%%%%%%%%%%%%%%%%%%%%%
% 9. Januar 2024
%%%%%%%%%%%%%%%%%%%%%%%%

\begin{satz}[Nullstellensatz für Polynome]
    \marginnote{[9. Jan]}
    \label{satz:nullstellen-polynome}
    Jedes nicht-triviale Polynom von Grad $n$ hat höchstens $n$ Nullstellen.

    \begin{proof}
        Induktion in $n$.~\\
        \begin{induktionsanfang}
            Für $n=0$ (konstantes Polynom) stimmt die Behauptung.
        \end{induktionsanfang}
        \begin{induktionsschritt}
            Angenommen die Behauptung stimmt für Polynom von Grad $n$. Sei $P$ Polynom von Grad $n+1$.\\
            1. Fall: $P$ hat keine Nullstelle $\impl$ Die Behauptung stimmt für $P$.\\
            2. Fall: $P(\zeta) = 0$ für ein $\zeta\in\R \annot{\impl}{\ref{satz:eigenschaften-polynome}} P(x) = (x-\zeta)\cdot Q\pair{x-\zeta}$. $Q$ ist Polynom von Grad $n$ mit höchstens $n$ Nullstellen. $\impl$ Anzahl der Nullstellen von $P$ ist $\leq n+1$.\qedhere
        \end{induktionsschritt}
    \end{proof}
\end{satz}

\begin{korollar}
    Sind $P,Q$ Polynome von Grad $\leq n$ und stimmen $P,Q$ an $n+1$ verschiedenen Stellen überein, so ist $P=Q$.

    \begin{proof}
        $h\definedas P-Q$ ist Polynom von Grad $\leq n$. Nach Satz~\ref{satz:nullstellen-polynome} hat $h$ damit höchstens $n$ Nullstellen, sofern $h$ nicht-trivial ist. Aber nach Voraussetzung existieren paarweise verschiedene $x_1,\dots, x_{n+1}$ mit
        \begin{alignat*}{2}
            P(x_j) &= Q(x_j)\quad&\forall j=1,\dots,n+1\\
            \impl h(x_j) &= 0\quad&\forall j=1,\dots,n+1
        \end{alignat*}
        Damit ist $h$ nach Satz~\ref{satz:nullstellen-polynome} trivial, das heißt $(\forall x\colon H(x) = 0) \impl (\forall x\colon P(x)=Q(x))$.
    \end{proof}
\end{korollar}

\begin{bemerkung}
    Der vorherige Beweis lässt sich auch auf analytische Polynome in den komplexen Zahlen übertragen.
\end{bemerkung}

\begin{korollar}[Koeffizientenvergleich]
    Zwei Polynome $P,Q$ sind genau dann gleich, wenn sie dieselben Koeffizienten haben.
\end{korollar}

\newpage


    \section{Cauchyprodukt und Exponentialfunktionen}
    \input{Kapitel/Cauchyprodukt}


    \section{Potenzreihen}
    \input{Kapitel/Potenzreihen}


    \section{Stetige Funktionen einer reellen (oder komplexen) Variablen}
    \input{Kapitel/Stetigkeit}


    \section{Der Zwischenwertsatz}
    \input{Kapitel/Zwischenwertsatz}


    \section{Der Satz von Weierstraß}
    \input{Kapitel/Satz_Weierstrass}


    \section{Grenzwerte von Funktionen}
    \input{Kapitel/Grenzwerte_Funktionen}


    \section{Gleichmäßige Stetigkeit und gleichmäßige Konvergenz}
    \subsection{Gleichmäßige und Lipschitz-Stetigkeit}
\thispagestyle{pagenumberonly}

\begin{definition}[Gleichmäßige Stetigkeit] % Def 1
    Sei $f: D\fromto \R$ (oder $\R^d$) und $D\sbset\K$. $f$ heißt gleichmäßig stetig auf $D$, falls
    \begin{align*}
        \fa\varepsilon > 0\ex\delta\colon \abs{f(x)-f(y)} < \varepsilon\quad\fa x,y\in D \text{ mit } \abs{x-y} < \delta
    \end{align*}
\end{definition}

\begin{bemerkung}
    Gleichmäßige Stetigkeit ist nach der Definition eine strengere Eigenschaft als Stetigkeit auf $D$. Das heißt jede gleichmäßig stetige Funktion ist auch stetig, aber nicht umgekehrt.
\end{bemerkung}

\begin{beispiel}
    \begin{align*}
        f: \R\fromto\R,~x\mapsto\frac{1}{1+x^2}
    \end{align*}
    ist gleichmäßig stetig. (Übung)
\end{beispiel}
\begin{beispiel}
    \begin{align*}
        f: \rinterv{0,1}\fromto \R,~x\mapsto \frac{1}{x}
    \end{align*}
    ist stetig, aber nicht gleichmäßig stetig.
    \begin{proof}
        Für $0 < x < y = 2x$ gilt
        \begin{align*}
            \abs{f(x)-f(y)} &= \abs{\frac{1}{x} - \frac{1}{y}} = \frac{\abs{y-x}}{xy} = \frac{1}{y}\geq 1\qedhere
        \end{align*}
    \end{proof}
\end{beispiel}

\begin{definition}[Lipschitz-Stetigkeit]
    Eine Funktion $f: D\fromto\R$ (oder $\R^d$) heißt Lipschitz-stetig, falls
    \begin{align*}
        \ex L\geq 0\colon \abs{f(x)-f(y)} \leq L\cdot\abs{x-y}\quad\forall x,y\in D
    \end{align*}
    Jede Lipschitz-stetige Funktion ist gleichmäßig stetig. $(\delta = \frac{\varepsilon}{L})$
\end{definition}

\begin{satz}[Heine, 1872] % Satz 3
    \label{satz:17-3}
    Sei $K\sbset\R$ kompakt und $f: K\fromto\R$ (oder $\R^d$) stetig. Dann ist $f$ gleichmäßig stetig.
    \begin{proof}
        Angenommen $f$ ist nicht gleichmäßig stetig.
        \begin{align*}
            \impl\ex\varepsilon > 0\fa \delta > 0\ex x,y\in &K\colon\abs{x-y} < \delta \text{ und } \abs{f(x)-f(y)} > \varepsilon
            \intertext{Wähle $\delta = \frac{1}{n}$}
            \impl\ex x_n, y_n\sbset K\colon \abs{x_n- y_n} &< \frac{1}{n} \text{ aber } \abs{f(x_n)-f(y_n)} \geq \varepsilon > 0\\
            \impl x_n - y_n &\fromto 0 \text{ für } n\fromto\infty
            \intertext{Da $K$ kompakt $\ex$Konvergente TF $(y_{n_l})_l$ von $(y_n)_n$ nach Satz~\ref{satz:bolzano-weierstrass}}
            y &\definedas \lim_{l\toinf} (y_{n_l}) \text{ existiert in } K
            \intertext{Für eine Teilfolge $(x_{n_l})_l$ von $(x_n)_n$ gilt}
            \abs{x_{n_l} - y} &= \abs{x_{n_l} - y_{n_l} + y_{n_l} - y}\\
            &\leq \underbrace{\abs{x_{n_l} - y_{n_l}}}_{<\frac{1}{n}\fromto 0} + \underbrace{\abs{y_{n_l} - y}}_{\fromto 0}\fromto 0\\
            \impl \abs{f(x_{n_l}) - f(y_{n_l})} &\geq \varepsilon > 0
            \intertext{Aber}
            \abs{f(x_{n_l}) - f(y_{n_l})} &= \abs{f(x_{n_l}) - f(y) + f(y) - f(y_{n_l})}\\
            &\leq \underbrace{\abs{f(x_{n_l}) - f(y)}}_{\fromto 0} + \underbrace{\abs{f(y)-f(y_{n_l})}}_{\fromto 0}
        \end{align*}
        Damit ergibt sich ein Widerspruch zur Stetigkeit von $f$ und $f$ ist damit gleichmäßig stetig.
    \end{proof}
\end{satz}

\subsection{Punktweise und gleichmäßige Konvergenz von Funktionenfolgen}

Wir betrachten Folgen von Funktionen. $f_n: D\fromto \R$ (oder $\R^d$) $\leadsto$ Folge $(f_n)_n$ von Funktionen.

\begin{definition}[Punktweise Konvergenz] % Def 4
    Eine Funktionenfolge $(f_n)_n$, $f_n: D\fromto\R$ (oder $\R^d$) konvergiert punktweise falls
    \begin{align*}
        \lim_{\ntoinf} f_n(x) \text{ existiert für jedes } x\in D
    \end{align*}
    Das heißt $(f_n(x))_n$ ist eine konvergente Folge $\fa x\in\R$. Dann definieren wir
    \begin{align*}
        f(x) \definedas \lim_{\ntoinf} f_n(x)
    \end{align*}
    eine Funktion $f: D\fromto \R$ (oder $\R^d$). Und sagen $f$ ist der punktweise Limes der Funktionenfolge $f_n(x) \fromto f(x)~\fa x\in D$.
\end{definition}

\begin{beispiel}
    Die Funktion
    \begin{align*}
        f_n(x) &= x^n\quad 0 \leq x \leq 1
        \intertext{konvergiert punktweise gegen}
        f(x) &= \begin{cases}
                    0\quad &0 \leq x < 1\\
                    1\quad &x = 1
        \end{cases}
    \end{align*}
\end{beispiel}

\begin{beispiel}
    Die Funktion
    \begin{align*}
        f_n(x) &= x^{\frac{1}{n}}\quad 0 \leq x \leq 1
        \intertext{ist stetig und punktweise konvergent gegen}
        f(x) &= \begin{cases}
                    0\quad&x = 0\\
                    1\quad&0 < x \leq 1
        \end{cases}
    \end{align*}
\end{beispiel}

\begin{beispiel}
    Die Funktion
    \begin{align*}
        f_n(x) &= \pair{1-x^2}^{\frac{n}{2}}\quad -1 \leq x \leq 1
        \intertext{ist stetig und punktweise konvergent gegen}
        f(x) &= \begin{cases}
                    1\quad&x = 0\\
                    0\quad&0 < \abs{x}\leq 1
        \end{cases}
    \end{align*}
\end{beispiel}

\begin{definition}[Gleichmäßige Konvergenz - Weierstraß 1841] % Definition 5
    $D\sbset\R$, Funktionenfolge $f_n: D\fromto\R$ (oder $\R^d$). $(f_n)_n$ konvergiert gleichmäßig gegen $f: D\fromto\R$ (oder $\R^d$) falls
    \begin{align*}
        \fa\varepsilon > 0\ex N\in \N \text{ mit } \abs{f_n(x) - f(x)} < \varepsilon\quad\fa n\geq N, x\in D
    \end{align*}
\end{definition}

\begin{bemerkung}
    Also gilt bei gleichmäßiger Konvergenz
    \begin{align*}
        \sup_{x\in D} \abs{f_n(x) - f(x)} &\leq \varepsilon\\
        \impl \lim_{\ntoinf} \sup_{x\in D}\abs{f_n(x) - f(x))} &= 0\\
        \equivalent \limsup_{\ntoinf} \pair{\abs{f_n(x) - f(x)}} &= 0
    \end{align*}
\end{bemerkung}

%%%%%%%%%%%%%%%%%%%%%%%%
% 01. Februar 2024
%%%%%%%%%%%%%%%%%%%%%%%%

\begin{notation}[Supremumsnorm]
    \marginnote{[01. Feb]}
    Es sei $f: D\fromto \R$ (oder $\R^d$, $\C$). Dann schreiben wir
    \begin{align*}
        \norm{f}_{\infty} &\definedas \norm{f}_{D,\infty} = \norm{f}_{L^{\infty}\of{D}}\\
        &= \sup_{x\in D} \abs{f(x)}
    \end{align*}
    Norm auf dem Vektorraum der beschränkten Funktionen auf $D$.
\end{notation}

\begin{satz}[Cauchy-Kriterium für gleichmäßige Konvergenz]
    Es sei $(f_n)_n$, $f_n: D\fromto \R$ (oder $\R^d$). Dann konvergiert $(f_n)_n$ genau dann gleichmäßig gegen $f$, wenn
    \begin{align*}
        \fa\varepsilon > 0\ex N\in\N\colon \abs{f_n(x) - f_m(x)} < \varepsilon\quad\forall x\in D, n,m\geq N
    \end{align*}
    \begin{proof}
        \anf{$\impl$}: $f(x) = \biglim{\ntoinf} f_n(x)$ existiert $\fa x\in D$. Dann gilt unabhängig von $x\in D$
        \begin{align*}
            \impl \abs{f_n(x) - f_m(x)} \leq \underbrace{\abs{f_n(x) - f(x)}}_{<\frac{\varepsilon}{2}} + \underbrace{\abs{f(x) - f_n(x)}}_{<\frac{\varepsilon}{2}} < \varepsilon\quad\fa n,m\geq N
        \end{align*}
        \anf{$\Leftarrow$}: Für $x\in D$ ist $(f_n(x))_n$ eine Cauchy-Folge. Und $f(x) = \biglim{\ntoinf} f_n(x)$ existiert $\fa x\in D$.
        \begin{align*}
            \abs{f_n(x) - f(x)} = \lim_{\ntoinf} \abs{f_n(x) - f_m(x)} &< \varepsilon\quad \forall n\geq N
            \intertext{Sei $\varepsilon > 0$}
            \impl \ex N\in\N\colon \abs{f_n(x) - f_m(x)} &< \varepsilon\quad\fa n,m\geq N\\
            \impl \abs{f_n(x) - f(x)} = \lim_{m\toinf} \abs{f_n(x) - f_m(x)} &< \varepsilon\quad\fa n\geq N\\
            \impl \sup_{x\in D} \abs{f_n(x) - f(x)} &\leq \varepsilon\quad\fa n\geq N\\
            \impl (f_n)_n \text{ geht gleichmäßig}& \text{  gegen } f\qedhere
        \end{align*}
    \end{proof}
\end{satz}

\begin{satz}[Weierstraß 1861] % Satz 7
    \label{satz:17-7}
    Seien $f_n: D \fromto \R$ (oder $\R^d$, $\C$) stetige Funktionen, welche gleichmäßig gegen eine Funktion $f$ konvergieren. Dann ist $f$ stetig!
    \begin{proof}
        Geg. $x_0\in D$, $x\in D$.
        \begin{align*}
            \abs{f(x) - f(x_0)} &= \abs{f(x) - f_n(x) + f_n(x) - f(x_0)}\\
            &\leq \abs{f(x) - f_n(x)} + \abs{f_n(x) - f_n(x_0)} + \abs{f_n(x) - f(x_0)}
            \intertext{Wir wenden den $\frac{\varepsilon}{3}$-Trick an}
            \fa\varepsilon > 0\ex N\in \N\colon \abs{f_n(y) - f(y)} &< \frac{\varepsilon}{3}\quad\forall y\in D, n\geq N
            \intertext{Wir fixieren $n=N$. Dann ist $f_n$ stetig}
            \impl \fa\varepsilon > 0\ex \delta > 0\colon \abs{f_n(x) - f_n(x_0)} &< \frac{\varepsilon}{3}\quad \text{ für } \abs{x-x_0} < \delta\\
            \intertext{Für $x\in D$, $\abs{x-x_0} < \delta$ gilt}
            \abs{f(x) - f(x_0)} &\leq \abs{f(x) - f_n(x)} + \abs{f_n(x) - f_n(x_0)} + \abs{f_n(x_0) - f(x_0)}\\
            &< \frac{\varepsilon}{3} + \frac{\varepsilon}{3} + \frac{\varepsilon}{3} = \varepsilon\\
            \impl f & \text{ ist stetig } \qedhere
        \end{align*}
    \end{proof}
\end{satz}

\begin{satz}[Weierstraß' M-Test] % Satz 8
    \label{satz:17-8}
    Eine Reihe $ \sum_{n=0}^{\infty} f_n$ von Funktionen $f_n: D\fromto \R$ (oder $\R^d$) konvergiert gleichmäßig, wenn sie eine konvergente Majorante hat, das heißt $\ex M_n\geq 0, N_0\in \N$ mit
    \begin{align*}
        \abs{f_n(x)} &\leq M_n\quad\forall x\in D, n\geq N_0
        \intertext{und}
        \sum_{n=0}^{\infty} M_n &< \infty
    \end{align*}
    \begin{proof}
        Partialsummen
        \begin{align*}
            s_n(x) &\definedas \sum_{j=0}^{n}  f_j(x)
            \intertext{Wir betrachten $n,m\geq N_0$}
            \abs{s_n(x) - s_m(x)} = \abs{\sum_{j=m+1}^{n} f_j(x)} &\leq \sum_{j=m+1}^{n} \underbrace{\abs{f_j(x)}}_{\leq M_j} \leq \sum_{j=m+1}^{n} M_j\\
            \impl \abs{s_n(x) - s_m(x)} &\leq \sum_{j=m+1}^{\infty} M_j \fromto 0\text{ für } m\fromto\infty
            \intertext{Haben}
            \impl \sup_{n\geq m} \sup_{x\in D} \abs{s_n(x) - s_m(x)} &\fromto 0 \text{ für } m\fromto\infty\\
            \equivalent s_{n} \text{ konvergiert} &\text{ gleichmäßig auf } D\qedhere
        \end{align*}
        Außerdem ist für $f_n$ stetig auch $s_n(x)$ stetig, da endliche Summen von stetigen Funktionen stetig sind. Und $s(x) = \biglim{\ntoinf} s_n(x)$ ist dann auch stetig nach Satz~\ref{satz:17-7}.
    \end{proof}
\end{satz}

\begin{anwendung}[Potenzreihen]
    Satz~\ref{satz:17-7} und Satz~\ref{satz:17-8} gelten auch für Funktionen $f_n: D\fromto \C$, $D\sbset \C$. Wir betrachten die Potenzreihe
    \begin{align*}
        f(x) &= \sum_{n=0}^{\infty} a_n x^n
        \intertext{und Partialsummen}
        s_n(x) &= \sum_{j=0}^{n} a_{j} x^j
        \intertext{Als Summe von Polynomen sind die Partialsummen stetig. Wir wollen Weierstraß' M-Test anwenden. Sei $R > 0$ Konvergenzradius der Potenzreihe}
        \impl \forall \abs{z} < R \text{ existiert } f(z) &= \sum_{n=0}^{\infty} a_n z^n
        \intertext{Geg. $\delta > 0, R-\delta > 0$ sei $z_1\in \C$, $\abs{z_1} = R - \frac{\delta}{2} < R$. Aus der Verbesserung von Lemma~\ref{lemma:temp-4} folgt}
        \ex M \geq 0\colon \abs{\sum_{n=k+1}^{\infty} a_n z^n} &\leq M\cdot \abs{\frac{z}{z_1}}^{k+1}\quad\fa \abs{z}< \abs{z_1} = R - \frac{\delta}{2}
        \intertext{Sei $\abs{z} \leq R- \delta$}
        \impl \frac{\abs{z}}{\abs{z_1}} &\leq \frac{R-\delta}{R-\frac{\delta}{2}} = q < 1\\
        \impl \abs{\sum_{n=k+1}^{\infty} a_n z^n} &\leq M \cdot q^{k+1}\\
        \impl \abs{s(z) - s_k(z)} &= \abs{\sum_{n=k+1}^{\infty} a_n z^n}\\
        &\leq M \cdot q^{k+1} \fromto 0 \text{ für } k\fromto \infty\\
        \sup_{\abs{z} \leq R - \delta} \abs{s(z) - s_k(z)} &\leq M \cdot q^{k+1} \fromto 0 \text{ für } k\fromto\infty
        \intertext{Das heißt die Partialsumme}
        s_k(z) &= \sum_{n=0}^{k} a_n z^n
        \intertext{ konvergiert gleichmäßig gegen $s(z)$ für alle $\abs{z} \leq R-\delta$.}
        \sum a_n z_1^n \text{ konvergent } &\impl a_n z_1^n \text{ Nullfolge }\\
        \impl M &= \sup_{n\geq 0} \abs{a_n z_1^n} < \infty\\
        \abs{a_n z^n} = \abs{a_n z_1^n\cdot\pair{\frac{z}{z_1}}^n} &\leq M \cdot \abs{\frac{z}{z_1}}^n \leq M \cdot q^n\\
        \abs{z} &\leq R-\delta
    \end{align*}
\end{anwendung}

\begin{satz}[Weierstraß] % Satz 9
    \label{satz:17-9}
    Sei $a <b$, $f: \interv{a,b}\fromto \R$ stetig. Dann gilt es existiert eine Folge von Polynomen $(P_n)_n$, welche gleichmäßig auf $\interv{a,b}$ gegen $f$ konvergiert. Das heißt
    \begin{align*}
        \lim_{n\fromto\infty} \norm{f-P_n}_{\infty}  = \biglim{\ntoinf} \sup_{a \leq x \leq b} \abs{f(x) - P_n(x)} = 0
    \end{align*}

    \begin{proof}
    \marginnote{[*]}
    (Der konstruktive Beweis für den Approximationssatz von Weierstraß mittels Bernstein-Polynomen, der in der Vorlesung behandelt wurde, fehlt hier).
    \end{proof}
\end{satz}

\newpage


    \section{Ableitung (engl. Differention)}
    %%%%%%%%%%%%%%%%%%%%%%%%
% 06. Februar 2024
%%%%%%%%%%%%%%%%%%%%%%%%

\thispagestyle{pagenumberonly}

\subsection{Ableitung als Grenzwert}

\begin{definition} % Definition 1
    \marginnote{[06. Feb]}
    Es seien $D = \pair{a,b}\sbset\R$, $x\in D$ und $f: D\fromto \R$. $f$ heißt im Punkt $x$ von rechts differenzierbar, falls
    \begin{align*}
        f'_+ &\definedas \lim_{h\fromto 0_+} \frac{f(x+h)-f(x)}{h}
        \intertext{existiert. $f$ ist von links differenzierbar, falls}
        f'_- &\definedas \lim_{h\fromto 0_-} \frac{f(x+h)-f(x)}{h}
    \end{align*}
    existiert. $f$ ist im Punkt $x$ differenzierbar, falls $f'_+$ und $f'_-$ existieren und $f'_+ = f'_-$. Das ist äquivalent zu der Existenz von
    \begin{align*}
        \lim_{\substack{h\fromto 0\\ h\neq 0}} \frac{f(x+h)-f(x)}{h}
    \end{align*}
\end{definition}

\begin{satz} % Satz 2
    \label{satz:18-2}
    Sei $a\in\pair{c,d}\sbset\R$. Die Funktion $f: \pair{c,d}\fromto\R$ ist genau dann im Punkt $a$ differenzierbar, wenn $\ex C\in\R$ mit $f(x) = f(a)+C\cdot\pair{x-a} + \varphi\of{x}$ wobei
    \begin{align*}
        \lim_{x\fromto a} \frac{\varphi\of{x}}{x-a} = 0
    \end{align*}
    \begin{proof}
        \anf{$\impl$}: Sei $f$ differenzierbar. Wir wählen $\varphi\of{x} \definedas f(x)-f(a) - f'(a)\cdot\pair{x-a}$ und $C=f'\of{a}$.
        \begin{align*}
            \lim_{x\fromto a} \frac{\varphi\of{x}}{x-a} &= \lim_{x\fromto a} \underbrace{\frac{f(x)-f(a)}{x-a}}_{\fromto f'(a)} - f'(a) = 0
        \end{align*}
        \anf{$\Leftarrow$}: Es gilt $f(x) = f(a)+C\cdot\pair{x-a} + \varphi\of{x}$ und $\frac{\varphi(x)}{x-a}\fromto 0$ für $x\fromto a$
        \begin{align*}
            &\impl \frac{f(x)-f(a)}{x-a} = C + \frac{\varphi(x)}{x-a}\\
            &\impl \abs{\frac{f(x)-f(a)}{x-a} - C} \fromto 0\\
            &\impl \frac{f(x)-f(a)}{x-a}\fromto C\\
            &\impl f \text{ ist in $a$ differenzierbar}\qedhere
        \end{align*}
    \end{proof}
\end{satz}

\begin{korollar}
    \theoremescape
    \label{korollar:abschaetzungen-ableitung}
    \begin{enumerate}[label=(\roman*)]
        \item Wenn $f$ im Punkt $a$ differenzierbar ist, dann ist $f$ im Punkt $a$ auch stetig.
        \item Sei $f'(a) \neq 0$. Dann gilt
        \begin{align*}
            \exists h_0~\forall h=x-a \text{ mit } \abs{h} < h_0,~h\neq 0\colon \abs{f(x)-f(a)}\geq\frac{1}{2}\abs{f'(a)}\cdot\abs{x-a}
        \end{align*}
        \newpage
        \item $\exists h_0$, so dass für alle $x$ mit $\abs{x-a} < h_0,~x\neq a$ gilt
        \begin{enumerate}[label=(\arabic*)]
            \item \fixedspace{6cm}{$\abs{f(x)-f(a)} \leq 2\abs{f'(a)}\cdot\abs{x-a}$} für $f'(a)\neq 0$
            \item \fixedspace{6cm}{$\abs{f(x)-f(a)} \leq \varepsilon\cdot\abs{x-a}$} für $f'(a) = 0$ \quad ($\forall\varepsilon > 0$, $h_0$ ist von $\varepsilon$ abhängig)
        \end{enumerate}
    \end{enumerate}

    \begin{proof}[Beweis (i)]
        Für $x\fromto a$ gilt
        \begin{align*}
            f(x)-f(a) &= C\cdot\underbrace{\pair{x-a}}_{\fromto 0} + \underbrace{\frac{\varphi(x)}{x-a}}_{\fromto 0}\cdot\underbrace{\pair{x-a}}_{\fromto 0}\fromto 0\qedhere
        \end{align*}
    \end{proof}

    \begin{proof}[Beweis (ii)]
        Für $x\fromto a$ gilt
        \begin{align*}
            f(x) &= f(a) + f'(a)\cdot\pair{x-a} + \frac{\varphi(x)}{x-a}\cdot\pair{x-a}\\
            \abs{f(x)-f(a)} &\geq \abs{f'(a)}\cdot\abs{x-a} - \underbrace{\abs{\frac{\varphi(x)}{x-a}}}_{\fromto 0}\cdot\abs{x-a}\geq \frac{1}{2}\cdot\abs{f'(a)}\cdot\abs{x-a}\qedhere
        \end{align*}
    \end{proof}
    \begin{proof}[Beweis (iii)]
    (1)
        Für $x\fromto a$ gilt
        \begin{align*}
            \abs{f(x)-f(a)} &\leq \abs{f'(a)}\cdot\abs{x-a} + \underbrace{\abs{\frac{\varphi(x)}{x-a}}}_{\fromto 0} \cdot \abs{x-a} \leq 2\abs{f'(a)}\cdot\abs{x-a}
            \intertext{(2)}
            \abs{f(x)-f(a)} &= \abs{\underbrace{f'(a)}_{\fromto 0} \cdot \pair{x-a} + \underbrace{\frac{\varphi(x)}{x-a}}_{\fromto 0}\cdot\pair{x-a}} \leq \varepsilon\cdot\abs{x-a}\qedhere
        \end{align*}
    \end{proof}
\end{korollar}

\subsection{Ableitungsregeln}

\begin{satz} % Satz 3
    \label{satz:ableitungsregeln}
    Seien $f, g: (a,b)\fromto \R$ differenzierbar, $\lambda\in\R$, $x\in\pair{a,b}$. Dann gilt
    \begin{enumerate}[label=(\roman*)]
        \item $\pair{f+g}'\of{x} = f'(x)+g'(x)$
        \item $\pair{\lambda\cdot f}'\of{x} = \lambda\cdot f'(x)$
        \item $\pair{f\cdot g}'\of{x} = f'(x)\cdot g(x) + g'(x)\cdot f(x)$\quad\quad(Produktregel)
    \end{enumerate}
    \begin{proof}[Beweis (iii)]
        \begin{align*}
            \pair{f\cdot g}'\of{x} &= \lim_{h\fromto 0} \frac{f(x+h)g(x+h) - f(x)g(x)}{h}\\
            &= \lim_{h\fromto 0} \frac{f(x+h)g(x+h) - f(x+h)g(x) + f(x+h)g(x) - f(x)g(x)}{h}\\
            &= \lim_{h\fromto 0} f(x+h)\cdot \underbrace{\frac{g(x+h)-g(x)}{h}}_{\fromto g'(x)} + \lim_{h\fromto 0} g(x) \cdot \underbrace{\frac{f(x+h)-f(x)}{h}}_{\fromto f'(x)}\\
            &= f(x) \cdot g'(x) + g(x)\cdot f'(x)\qedhere
        \end{align*}
    \end{proof}
\end{satz}

\begin{uebung}
    Beweisen Sie die verbleibenden Aussagen des vorherigen Satzes.
\end{uebung}

\begin{satz}[Kettenregel] % Satz 4
    \label{satz:kettenregel}
    Seien $f: \pair{a,b}\fromto \R$ und $g: (c,d)\fromto \R$ Funktionen mit $f\interv{\pair{a,b}} \sbset \pair{c,d}$. Die Funktion $f$ sei im Punkt $x\in\pair{a,b}$ differenzierbar und $g$ sei im Punkt $y\definedas f(x)$ differenzierbar. Dann gilt
    \begin{align*}
        \pair{g\circ f}'(x) = g'(f(x)) \cdot f'(x)
    \end{align*}
    \begin{proof}
        Wir definieren $F(x) \definedas g(f(x))$ und unterscheiden zwei Fälle.\\
        \textsc{Fall 1.} $f'(x) \neq 0$. Nach Korollar~\ref{korollar:abschaetzungen-ableitung} gilt
        \begin{align*}
            \ex h_0, \abs{h} < h_0\colon \abs{f(x+h)-f(x)} &\geq \frac{1}{2}\abs{f'(x)}\abs{h} \neq 0\\
            \lim_{h\fromto 0} \frac{F(x+h)-F(x)}{h} &= \lim_{h\fromto 0} \frac{\pair{F(x+h) - F(x)}\cdot\pair{f(x+h)-f(x)}}{\pair{f(x+h)-f(x)}\cdot h}\\
            &= \lim_{h\fromto 0} \frac{g(f(x+h))-g(f(x))}{f(x+h)-f(x)} \cdot \underbrace{\frac{f(x+h)-f(x)}{h}}_{\fromto f'(x)}
            \intertext{$f(x)=y$ und $f(x+h) = y + \Delta y$}
            \lim_{h\fromto 0} \frac{g(f(x+h))-g(f(x))}{f(x+h)-f(x)} &= \lim_{h\fromto 0} \frac{g(y+\Delta y) - g(y)}{\Delta y}\tag{$\Delta y \neq 0$}\\
            &= \lim_{\Delta y \fromto 0} \frac{g(y+\Delta y) - g(y)}{\Delta y}\\
            &= g'(y) = g'(f(x))\\
            \impl \frac{F(x+h)-F(x)}{h} &\fromto g'(f(x)) \cdot f'(x)
            \intertext{\textsc{Fall 2.} $f'(x) = 0$. Dann gilt nach Korollar~\ref{korollar:abschaetzungen-ableitung}}
            \abs{\frac{g(f(x+h)) - g(f(x))}{h}} &\leq \frac{c\cdot\abs{f(x+h)-f(x)}}{h}\\
            &\leq \frac{c\cdot\varepsilon\cdot \abs{x+h-x}}{h}\\
            &= c\cdot\varepsilon\\
            \intertext{mit $c=\max\set{2\cdot\abs{g'(f(x))}, 1}$ und $\varepsilon$ beliebig klein}
            \impl \lim_{h\fromto 0} \frac{F(x+h)-F(x)}{h} = 0 &= g'\of{f\of{x}} \cdot 0 = g'\of{f\of{x}}\cdot f'\of{x}\\
            \intertext{Damit folgt insgesamt}
            \pair{g\circ f}'(x) &= g'(f(x))\cdot f'(x)\qedhere
        \end{align*}
    \end{proof}
\end{satz}

\begin{satz}[Quotientenregel] % Satz 5
    \label{satz:quotient-ableitung}
    Für zwei differenzierbare Funktionen $u,v$ gilt
    \begin{align*}
        \pair{\frac{v}{u}}' = \frac{v'\cdot u - u'\cdot v}{u^2}
    \end{align*}
    \begin{proof}
        \begin{align*}
            \pair{\frac{v}{u}}' &= \pair{v\cdot\frac{1}{u}}' = v'\cdot \frac{1}{u} + v\cdot\frac{1}{u^2}\cdot u'\cdot (-1) = \frac{v'u - u'v}{u^2}\qedhere
        \end{align*}
    \end{proof}
\end{satz}

\begin{beispiel}[Ableitung der Exponentialfunktion]
    Es sei $a\in\C$, $x\in\R$. Wir wollen $e^{ax}$ ableiten
    \begin{align*}
        \pair{e^{ax}}' &= \lim_{h\fromto 0} \frac{e^{a(x+h)} - e^{ax}}{h} = \lim_{h\fromto 0}e^{ax} \cdot \frac{e^{ah}-1}{h} = e^{ax}\cdot \lim_{h\fromto 0} \frac{e^{ah}-1}{h}\\
        e^{ah} &= \sum_{n=0}^{\infty} \frac{(ah)^n}{n!} = 1 + ah + (ah)^2\cdot\sum_{n=2}^{\infty} \frac{(ah)^{n-2}}{n!}
        \intertext{Wir schätzen den letzten Teil der Gleichung mit $\abs{h} < \frac{1}{\abs{a}}$ ab}
        \abs{ \sum_{n=2}^{\infty} \frac{(ah)^{n-2}}{n!}} &\leq \sum_{n=2}^{\infty} \frac{1}{n!} < e\\
        \impl \pair{e^{ax}}' &= e^{ax}\cdot \lim_{h\fromto 0} \frac{1+ah+(ah)^2\cdot e - 1}{h} = a\cdot e^{ax}\tag{$a\in\C$}
    \end{align*}
\end{beispiel}

%%%%%%%%%%%%%%%%%%%%%%%%
% 08. Februar 2024
%%%%%%%%%%%%%%%%%%%%%%%%

\begin{beispiel}[Ableitung von $\sin$ und $\cos$]
    \marginnote{[08. Feb]}
    \begin{align*}
        \pair{\sin x}' &= \pair{\Im e^{ix}}' = \Im \pair{e^{ix}}' = \Im\pair{i\cdot e^{ix}} = \Im\of{i\cdot\pair{\cos x + i\cdot \sin x}} = \cos x
        \intertext{Analog lässt sich zeigen, dass gilt}
        \pair{\cos x}' &= -\sin x
    \end{align*}
\end{beispiel}
\begin{beispiel}[Ableitung von $\tan$]
    \begin{align*}
        \pair{\tan x}' &= \pair{\frac{\sin x}{\cos x}}' = \frac{\cos x \cdot \cos x -\pair{-\sin x \cdot \sin x}}{\cos^2 x} = \frac{\cos^2 x + \sin^2 x}{\cos^2 x} = \frac{1}{\cos^2 x}
    \end{align*}
\end{beispiel}

\begin{satz}[Ableitung der Umkehrfunktion] % Satz 6
    \label{satz:ableitung-umkehrfunktion}
    Sei $f: \pair{a,b} \fromto \pair{c,d}$ eine bijektive Abbildung, die im Punkt $x\in\pair{a,b}$ differenzierbar ist. Dann ist die Funktion $f^{-1}$ an der Stelle $y=f\of{x}$ differenzierbar und es gilt
    \begin{align*}
        \pair{f^{-1}}'\of{y} = \frac{1}{f'\of{x}}
    \end{align*}
    \begin{proof}
        \begin{align*}
            \pair{f^{-1}}'\of{y} &= \lim_{h\fromto 0} \frac{\overbrace{f^{-1}\of{y+h}}^{\definedasbackwards \xi} - f^{-1}\of{y}}{h} = \lim_{h\fromto 0}\frac{\xi - x}{f\of{\xi} - f\of{x}}\\
            &= \lim_{h\fromto 0} \frac{1}{\frac{f\of{\xi} - f\of{x}}{\xi-x}} = \frac{1}{\lim_{h\fromto 0} \frac{f\of{\xi} - f\of{x}}{\xi-x}}
            \intertext{Wegen der Monotonie und Stetigkeit von $f$ können wir umformen zu}
            &= \frac{1}{\lim_{\xi\fromto x} \frac{f\of{\xi} - f\of{x}}{\xi-x}} = \frac{1}{f'\of{x}}\qedhere
        \end{align*}
    \end{proof}
\end{satz}

\begin{beispiel}[Ableitung des Logarithmus]
    Wir definieren $f\definedas e^x$. Damit gilt
    \begin{align*}
        \log y &= f^{-1}\of{y}\\
        \log\pair{e^x} &= x\\
        \pair{\log y}' &= \frac{1}{\pair{e^x}'_{\lvert y=e^x}} = \frac{1}{\pair{e^x}_{\lvert y=e^x}} = \frac{1}{y}
    \end{align*}
\end{beispiel}

\begin{bemerkung}[Ableitung von Potenzen]
    \footnote{In VL erst im nächsten Unterkapitel behandelt}
    \begin{align*}
        \pair{x^{\alpha}}' &= \pair{e^{\log x^{\alpha}}} = \pair{e^{\alpha\log x}}'
        \intertext{Nach Satz~\ref{satz:kettenregel}}
        &= e^{\alpha \log x} \cdot \alpha \cdot \pair{\log x}' = \frac{\alpha x^{\alpha}}{x} = \alpha \cdot x^{\alpha-1}
    \end{align*}
\end{bemerkung}

\begin{beispiel}[Ableitung von $\arccos$]
    Es sei $y\in\pair{-1, 1}$. Dann gilt
    \begin{align*}
        \pair{\arccos y}' &= \frac{1}{\pair{\cos x}'_{\lvert \cos x = y}} = \frac{1}{-\sin\of{\arccos\of{y}}}\\
        &= \frac{1}{\pm\sqrt{1-\cos^2\of{\arccos y}}} = \pm \frac{1}{\sqrt{1-y^2}}
    \end{align*}
\end{beispiel}

\subsection{Lokale Extrema und Mittelwertsätze}

\begin{definition}[Lokales Maximum und Minimum] % Definition 7
    Sei $f: \pair{a,b} \fromto\R$. Man sagt $f$ habe in $x\in\pair{a,b}$ ein lokales Maximum (Minimum), wenn ein $\varepsilon > 0$ existiert, so dass
    \begin{align*}
        f\of{x} \underset{(\leq)}{\geq} f\of{\xi}\quad \forall \xi\in\pair{x-\varepsilon, x+\varepsilon}
    \end{align*}
\end{definition}

\begin{satz}[Ableitung bei lokalen Extrema]
    \label{satz:ableitung-extrem}
    Sei $f: \pair{a,b} \fromto \R$ differenzierbar und $x\in\pair{a,b}$ ein lokales Extremum. Dann gilt $f'\of{x} = 0$.
    \begin{proof}
        $\ex \varepsilon > 0$, $f\of{\xi} \leq f\of{x}$, $\xi\in\pair{x-\varepsilon, x+\varepsilon}$
        \begin{align*}
            f'_{+}\of{x} &= \lim_{\substack{\xi\fromto x\\ \xi > x}} \frac{f\of{\xi} - f\of{x}}{\xi - x} \leq 0\\
            f'_{-}\of{x} &= \lim_{\substack{\xi\fromto x\\ \xi < x}} \frac{f\of{\xi} - f\of{x}}{\xi - x} \geq 0\\
            \impl f'_{+} &= f'_{-} = 0\qedhere
        \end{align*}
    \end{proof}
\end{satz}

\begin{bemerkung}
    Die Umkehrung gilt nicht. Wir betrachten $f: x\mapsto x^3$ mit $f'(0) = 3x^2 = 0$, aber die Funktion hat an der Stelle $x=0$ kein lokales Maximum oder Minimum.
\end{bemerkung}

\begin{bemerkung}
    Sei $f$ auf $\interv{a,b}$ stetig und auf $\pair{a,b}$ differenzierbar. Dann kann das Maximum/Minimum der Funktion auch auf den Intervall-Grenzen $a$ und $b$ liegen, obwohl die Ableitung für diese nicht bestimmbar ist.
\end{bemerkung}

\begin{satz}[Satz von Rolle]
    \label{satz:von-rolle}
    Sei $a < b$, $f: \interv{a,b} \fromto \R$ eine stetige Funktion mit $f(a) = f(b)$. Die Funktion $f$ sei in $\pair{a,b}$ differenzierbar. Dann $\ex\xi\in\pair{a,b}$ mit $f'\of{\xi} = 0$.
    \begin{proof}
        ~\\
        \textsc{Fall 1.} $f$ ist eine konstante Funktion. Dann gilt $f' = 0$.\\
        \textsc{Fall 2.} $f$ ist keine konstante Funktion. Das heißt nach Satz~\ref{satz:weierstrass-maximum-minimum} $\exists x$ mit $f\of{x} \neq f\of{a}$. Wenn $f(x) > f(a)$, dann existiert ein lokales Maximum bei $x_0\in\pair{a,b}$ und wenn $f(x) < f(a)$, dann existiert ein lokales Minimum bei $x_0\in\pair{a,b}$. Damit ist $f'\of{x_0} = 0$.
    \end{proof}
\end{satz}

\newpage

\begin{satz}[Mittelwertsatz]
    \label{satz:mittelwertsatz}
    Sei $a < b$, $f:\interv{a,b}\fromto\R$ stetig und in $\pair{a,b}$ differenzierbar. Dann gilt
    \begin{align*}
        \ex\xi\in\pair{a,b}\colon f\of{b} - f\of{a} = f'\of{\xi}\cdot\pair{b-a}
    \end{align*}
    \begin{proof}
        \begin{align*}
            F(x) &\definedas f(x) - \frac{f\of{b}-f\of{a}}{b-a}\cdot\pair{x-a}\\
            F(a) &= f(a) - \frac{f(b)-f(a)}{b-a}\cdot 0 = f(a)\\
            F(b) &= f(b) - \frac{f(b)-f(a)}{b-a}\cdot\pair{b-a} = f\of{a}\\
            \impl F(a) &= F(b)
            \intertext{Damit gilt nach Satz~\ref{satz:von-rolle}}
            \impl \ex\xi \text{ mit } F'\of{\xi} &= 0\\
            \impl f'\of{\xi} - \frac{f\of{b} - f\of{a}}{b-a} &= 0\\
            \impl f(b) - f(a) &= f'\of{\xi}\cdot\pair{b-a}\qedhere
        \end{align*}
    \end{proof}
\end{satz}

\begin{visualisierung}[Geometrische Anschauung des Mittelwertsatzes]
    Der Mittelwertsatz sagt aus, dass es einen Punkt auf jeder differenzierbaren Funktion gibt, dessen Tangente parallel zu einer affin-linearen Funktion durch $f(a)$ und $f(b)$ läuft.
    \begin{figure}[H]
        \centering
        \begin{tikzpicture}
            \draw[->] (0, 0) -- (4, 0);
            \draw[->] (0, 0) -- (0, 4);
            \draw (0, 0.1) -- (0, -0.1) node[below] {$a$};
            \draw (4.21*.6, 0.1) -- (4.21*.6, -0.1) node[below] {$\xi$};
            \draw (5*.6, 0.1) -- (5*.6, -0.1) node[below] {$b$};
            \fill (0,0) circle[radius=1.5pt];
            \fill (5*.6,6*0.6*0.6) circle[radius=1.5pt];
            \draw[scale=0.6, domain=0:5, smooth, variable=\x] plot ({\x}, {(-43/120 *\x*\x*\x*\x+71/20*\x*\x*\x-1337/120*\x*\x+259/20*\x)*0.6});
            \draw[scale=0.6, domain=-1:6, smooth, variable=\x] plot ({\x}, {(\x*6/5)*0.6});
            \draw[scale=0.6, domain=-1:6, smooth, variable=\x, dashed] plot ({\x}, {(\x*6/5+4.35)*0.6});
        \end{tikzpicture}
        \caption{Tangente an der Stelle $x=\xi$\\parallel zur Geraden durch die beiden Punkte}
    \end{figure}
\end{visualisierung}

\begin{korollar} % Korollar 10
    Es gilt genau dann $f'(x) = 0$ für alle $x\in\pair{a,b}$, wenn $f(x)$ eine konstante Funktion in $\interv{a,b}$ ist.
    \begin{proof}
        \begin{align*}
            f(x_1) - f(x_2) \annot[{&}]{=}{\ref{satz:mittelwertsatz}} f'\of{\xi}\cdot\pair{x_1 - x_2}\\
            &= 0\cdot\pair{x_1 - x_2}\\
            \impl f(x_1) &= f(x_2)\qedhere
        \end{align*}
    \end{proof}
\end{korollar}

\begin{satz} % Satz 11
    \label{satz:18-11}
    Seien $f$ und $g$ stetig auf $\interv{a,b}$ und auf $\pair{a,b}$ differenzierbar. Dann gilt
    \begin{align*}
        \ex\xi\in\pair{a,b}\colon \interv{f(b)-f(a)}\cdot g'\of{\xi} &= \interv{g(b)-g(a)}\cdot f'\of{\xi}
    \end{align*}
    \begin{proof}
        \begin{align*}
            h(t) &\definedas \interv{f(b)-f(a)}\cdot g(t) - \interv{g(b)-g(a)}\cdot f(t)\\
            \impl h(a) &= h(b)\\
            \annot{\impl}{\ref{satz:mittelwertsatz}} \ex \xi \in\pair{a,b} &\text{ mit } h'\of{\xi} = 0\\
            \impl \interv{f(b)-f(a)}\cdot g'\of{\xi} &- \interv{g(b)-g(a)}\cdot f'\of{\xi} = 0\\
            \impl \interv{f(b)-f(a)}\cdot g'\of{\xi} &= \interv{g(b)-g(a)}\cdot f'\of{\xi}\qedhere
        \end{align*}
    \end{proof}
\end{satz}

\begin{notation}[Ableitungen höherer Ordnung] % Definition 11
    Wir definieren für die zweite Ableitung
    \begin{align*}
        f'' &= \pair{f'}'
        \intertext{und allgemein}
        f^{(n)} &= \pair{f^{(n-1)}}'
    \end{align*}
\end{notation}

\begin{satz}[Der Taylorsche Satz] % Satz 12
    \label{satz:taylor}
    Es sei $f$ eine reelle Funktion auf $\interv{a,b}$ und $f^{(n-1)}$ sei stetig auf $\interv{a,b}$ und $f^{(n)}$ existiere auf $\pair{a,b}$. Sei $\pair{\alpha,\beta}\subseteq \interv{a,b}$ und
    \begin{align*}
        P_{n-1}(t) &= \sum_{k=0}^{n-1} \frac{f^{(k)}\of{\alpha}}{k!}\cdot\pair{t-\alpha}^k\tag{\footnotemark}
        \intertext{Dann $\ex\xi\in\pair{\alpha,\beta}$ so dass}
        f\of{\beta} &= P_{n-1}\of{\beta} + \frac{f^{(n)}\of{\xi}}{n!}\cdot\pair{\beta-\alpha}^n
    \end{align*}

    \footnotetext{In manchen Lehrbüchern wird auch $T_n\of{f,\alpha}$ als alternative Schreibweise zu $P_{n}\of{t}$ verwendet.}

    %%%%%%%%%%%%%%%%%%%%%%%%
    % 13. Februar 2023
    %%%%%%%%%%%%%%%%%%%%%%%%

    \begin{proof}
        \marginnote{[13. Feb]}
        \begin{align*}
            M&\definedas \frac{f(\beta)-P_{n-1}\of{\beta}}{\pair{\beta-\alpha}^n}\tag{$M\in\R$}\\
            g(t) &\definedas f(t)-P_{n-1}\of{t} - M \cdot \pair{t-\alpha}^n\tag{$\alpha\leq t\leq\beta$}\\
            g\of{\alpha} &= f\of{\alpha} - \overbrace{0 - f\of{\alpha}}^{P_{n-1}\of{\alpha}} - 0 = 0\\
            g'\of{\alpha} &= f'\of{\alpha} - f'\of{\alpha} = 0\\
            \vdots&\\
            g^{(n-1)}\of{\alpha} &= 0\\[10pt]
            g\of{\beta} &= f\of{\beta} - P_{n-1}\of{\beta} - \frac{f\of{\beta} - P_{n-1}\of{\beta}}{\pair{\beta-\alpha}^n}\cdot \pair{\beta-\alpha}^n = 0
            \intertext{1. Schritt}
            g\of{\alpha} &= 0,~g\of{\beta} = 0
            \intertext{Nach Satz~\ref{satz:von-rolle}}
            \impl \ex x_1\in\pair{\alpha, \beta}\colon g'(x_1) &= 0
            \intertext{2. Schritt}
            g'\of{\alpha} &= 0,~g'\of{x_1} = 0\\
            \impl \ex x_2\in\pair{\alpha, x_1}\colon g''(x_2) &= 0\\[10pt]
            g''(\alpha) &= 0,~g''\of{x_2} = 0\\
            \vdots\\
            \impl\ex x_n \in\pair{\alpha, x_{n-1}}\colon g^{(n)}\of{x_n} &= 0\tag{$\xi\definedas x_n$}\\
            \impl g^{(n)}\of{\xi} = f^{(n)}\of{\xi} - 0 - M\cdot n! &= 0\\
            \equivalent M = \frac{f^{(n)}\of{\xi}}{n!} &= \frac{f\of{\beta}-P_{n-1}\of{\beta}}{\pair{\beta-\alpha}^n}\\
            \equivalent f\of{\beta} - P_{n-1}\of{\beta} &= \frac{f^{(n)}\of{\xi}}{n!}\cdot\pair{\beta-\alpha}^n\\
            f\of{\beta} &= P_{n-1}\of{\beta} + \frac{f^{(n)}\of{\beta}}{n!}\pair{\beta-\alpha}^n\qedhere
        \end{align*}
    \end{proof}
\end{satz}

\begin{bemerkung}
    Der Taylorsche Satz ist eine Verallgemeinerung des Mittelwertsatzes für höhere Ableitungen.
\end{bemerkung}

\begin{korollar}[Monotonie] % Korollar 14
    \label{korollar:monotonie}
    Sei $f$ auf $\pair{a,b}$ differenzierbar mit $f' > 0$ ($f' < 0$). Dann ist $f$ streng monoton wachsend (fallend) auf $\pair{a,b}$.
    \begin{proof}
        Seien $x_1, x_2\in\pair{a,b}$ mit $x_2 > x_1$. Nach Satz~\ref{satz:mittelwertsatz} gilt
        \begin{align*}
            f(x_2) - f(x_1) &= \underbrace{f'(\xi)}_{> 0}\cdot\underbrace{\pair{x_2 - x_1}}_{> 0} > 0\qedhere
        \end{align*}
        Der Beweis für fallende Funktionen funktioniert analog.
    \end{proof}
\end{korollar}

\begin{bemerkung}[Über Max und Min]
    Es seien $f$ differenzierbare Funktion und $x_0$ lokales Extremum von $f$ und sei $f''\of{x_0}$ existent und positiv (negativ). Dann ist $x_0$ ein lokales Minimum (Maximum).

    \begin{proof}
        \begin{align*}
            f''\of{x_0} &= \lim_{\xi\fromto x_0} \frac{f'\of{\xi} - f'\of{x_0}}{\xi-x_0} > 0\\
            &\impl \ex\varepsilon > 0\colon \frac{f'\of{\xi} - f'\of{x_0}}{\xi-x_0} > 0\quad\forall\xi, 0 < \abs{x_0-\xi} < \varepsilon\\
            &\impl \begin{cases}
                       f'\of{\xi} < 0 \text{ für } \xi < x_0~\leadsto \text{ Funktion fällt}\\
                       f'\of{\xi} > 0 \text{ für } \xi > x_0~\leadsto \text{ Funktion steigt}
            \end{cases}
        \end{align*}
        Da die Funktion vor $x_0$ fällt und danach steigt, ist $x_0$ ein lokales Minimum.
    \end{proof}
\end{bemerkung}

\newpage

\subsection{Die Regel von l'Hospital}

\begin{satz}[Regel von de l'Hospital] % Satz 16
    \label{satz:l-hospital}
    Seien $f$ und $g$ differenzierbar in $\pair{a,b}$. Sei ferner $g'\of{x} \neq 0$ für alle $x\in\pair{a,b}$ und es gelte
    \begin{align*}
        \frac{f'(x)}{g'(x)} &\fromto A \text{ für } x \fromto a\tag{1}
        \intertext{Außerdem gelte}
        f(x) \fromto 0 \text{ und } g(x)&\fromto 0 \text{ für } x\fromto a\tag{2.1}
        \intertext{\underline{oder}}
        g(x)&\fromto \infty \text{ für } x\fromto a\tag{2.2}
        \intertext{Dann gilt}
        \frac{f(x)}{g(x)}&\fromto A \text{ für } x\fromto a
    \end{align*}
    Die analoge Behauptung ist wahr für $x\fromto b$ oder für $g(x)\fromto -\infty$. Außerdem ist $g'(a) = 0$ bei der Anwendung des Satzes erlaubt.

    \begin{proof}
        Sei $A < \infty \impl\ex q> A$. Sei $A < r < q$
        \begin{align*}
            \impl \frac{f'(x)}{g'(x)} &< r \text{ für } a < x < a + \varepsilon_0
            \intertext{Nach Satz~\ref{satz:18-11} gilt mit $a < x < y < a + \varepsilon_0$}
            \frac{f(x)-f(y)}{g(x)-g(y)} &= \frac{f'(\xi)}{g'(\xi)}  < r
            \intertext{\textsc{Fall 1.} $f(x)\fromto 0$, $g(x)\fromto 0$ für $x\fromto a$}
            \impl g(y) \neq 0, \text{ weil } g(a) &= 0,~g'(a)\neq 0\\
            \lim_{x\fromto a} \frac{\overbrace{f(x)}^{\fromto 0}-f(y)}{\underbrace{g(x)}_{\fromto 0}-g(y)} &= \lim_{x\fromto a} \frac{f(y)}{g(y)} \leq r < q\\
            \impl \fa a < y < a + \varepsilon_0\colon \frac{f(y)}{g(y)} &< q\\
            \impl \lim_{y\fromto a} \frac{f(y)}{g(y)} &< q
            \intertext{\textsc{Fall 2.} $g(x)\fromto\infty$ für $x\fromto a$. Dann gilt mit $a < x < y < a + \varepsilon_0$}
            \frac{f(x)-f(y)}{g(x)-g(y)} &< r\\
            \impl \frac{\frac{f(x)}{g(x)}- \frac{f(y)}{g(x)}}{1 - \frac{g(y)}{g(x)}} &< r\\
            \impl \frac{\frac{f(x)}{g(x)}}{1} &\leq r < q\\
            \impl\fa q > A\colon \frac{f(x)}{g(x)} &< q\tag{$a<x<a+\varepsilon_0$}
        \end{align*}
        \newpage
        \noindent In beiden Fällen ist der Quotient durch $q$ nach oben beschränkt. Sei $A > -\infty$
        \begin{align*}
            \impl\exists p\colon p &< r < A\\
            \impl \frac{f(x)-f(y)}{g(x)-g(y)} &= \frac{f'\of{\xi}}{g'\of{xi}} > r > p \tag{$a < x < y < a + \varepsilon_0$}
            \intertext{Mit der gleichen Argumentation wie davor ergibt sich}
            \frac{f(x)}{g(x)} &> p \impl \frac{f(x)}{g(x)} \fromto A&&\qedhere
        \end{align*}
    \end{proof}
\end{satz}

\begin{beispiel}
    \begin{align*}
        \lim_{x\fromto 1_-} \interv{\ln x \cdot \ln\of{1-x}} &= \lim_{x\fromto 1_-} \frac{\ln\pair{1-x}}{\frac{1}{\ln\of{x}}} \annot{=}{\ref{satz:l-hospital}} \lim_{x\fromto 1_-} \frac{\frac{1}{1-x}\cdot\pair{-1}}{-\frac{1}{\ln^2\of{x}} \cdot \frac{1}{x}}\\[4pt]
        &= \lim_{x\fromto 1_-} \frac{x\cdot\ln^2\of{x}}{1-x} = \lim_{x\fromto 1_-} \frac{\ln^2\of{x}}{1-x}\\[4pt]
        \annot[{&}]{=}{\ref{satz:l-hospital}} \lim_{x\fromto 1_-} \frac{2\ln\of{x}\cdot \frac{1}{x}}{-1} = 0
    \end{align*}
\end{beispiel}

\newpage


    \section{Konvexität}
    %%%%%%%%%%%%%%%%%%%%%%%%
% 15. Februar 2024
%%%%%%%%%%%%%%%%%%%%%%%%

\thispagestyle{pagenumberonly}

\subsection{Konvexe und konkave Funktionen}

\begin{skizze}[Konvexe Funktion]
    \marginnote{[15. Feb]}
    Wähle ein $\lambda\in\pair{0,1}$ und formuliere die Interpolation $r=\lambda x + \pair{1-\lambda}\cdot y$.
    \begin{figure}[H]
        \centering
        \begin{tikzpicture}
            \draw[->] (-1, 0) -- (3, 0);
            \draw[->] (0, -1) -- (0, 4);
            \draw (-0.95*0.5, 0.1) -- (-0.95*0.5, -0.1) node[below] {$x$};
            \draw (5.2*0.5, 0.1) -- (5.2*0.5, -0.1) node[below] {$y$};

            \fill (-0.95*.5,0.7875*.5) circle[radius=1.5pt] node[left] {$f(x)$};
            \fill (5.2*.5,5.4*.5) circle[radius=1.5pt] node[right] {$f(y)$};
            \draw[domain=-2:6, smooth, variable=\x] plot ({0.5*\x}, {(0.2*(\x-0.25)^2+0.5)*0.5}) node[anchor=east] {$f$};
            \draw[domain=-0.95:5.2, smooth, variable=\x, dashed] plot ({0.5*\x}, {(0.75*\x+1.5)*0.5});
        \end{tikzpicture}
        \caption{Konvexe Funktion mit eingezeichneter Sekante}
    \end{figure}
    \noindent Wir erhalten die Sekantengleichung $\lambda f(x) +\pair{1-\lambda}\cdot f(y)$. Die Funktion ist konvex, wenn sie unter der Sekante verläuft. Das heißt
    \begin{align*}
        \lambda f(x) +\pair{1-\lambda} f(y) \geq f\of{x+\pair{1-\lambda}\cdot y}\quad\forall\lambda\in\pair{0,1} \text{ und } x,y\in D
    \end{align*}
\end{skizze}

\begin{definition}[Konvexität/Konkavität]
    Sei $f: D\fromto \R$ wobei $D\subseteq \R$ ein Intervall. Die Funktion $f$ heißt konvex (konkav), falls für alle $x,y\in D$ und alle $\lambda \in\pair{0,1}$ gilt
    \begin{align*}
        f\of{\lambda x + \pair{1-\lambda}\cdot y} \underset{(\geq)}{\leq} \lambda f(x) + \pair{1-\lambda}\cdot f(y)
    \end{align*}
\end{definition}

\begin{beispiel}
    Wir betrachten $f: \R\fromto\R$ mit $x\mapsto x^2$. Sei $x,y\in\R$ und \OBDA sei $y\geq x$, $\lambda\in\pair{0,1}$. Dann ist zu zeigen
    \begin{align*}
        f\of{\lambda x + \pair{1-\lambda}\cdot y} &\leq \lambda f\of{x} + \pair{1-\lambda}\cdot  f\of{y}\\
        \equivalent \pair{\lambda x+\pair{1-\lambda}\cdot y}^2 &\leq \lambda x^{2} + \pair{1-\lambda}\cdot y^2\\
        \equivalent \lambda^2 x^2 + 2\lambda\pair{1-\lambda}\cdot xy + \pair{1-\lambda}^{2}\cdot y^2 & \leq \lambda x^2 + \pair{1-\lambda}\cdot y^2\\
        \equivalent \lambda\cdot\pair{\lambda -1}\cdot x^2 + \pair{1-\lambda}\cdot\pair{1-\lambda-1}\cdot y^2 + \lambda\pair{1-\lambda}\cdot 2xy &\leq 0\\
        \equivalent \lambda\pair{\lambda -1}\cdot\interv{x^2+y^2-2xy} &\leq 0\\
        \equivalent x^2+y^2-2xy &\geq 0\\
        \equivalent \pair{x-y}^2 &\geq 0
    \end{align*}
\end{beispiel}

\hfill

\begin{satz}[Konvexität und zweite Ableitung] % Satz 1
    Sei $D\subseteq\R$ ein offenes Intervall und $f: D\fromto\R$ eine auf $D$ zweimal differenzierbare Abbildung. Dann gilt
    \begin{align*}
        f \text{ ist konvex }\quad \equivalent\quad \forall x\in D\colon f''(x)\geq 0
    \end{align*}
    \newpage
    \begin{proof}
        \anf{$\Leftarrow$}: Da $f'' > 0$ auf $D$, ist $f'$ auf $D$ monoton wachsend. Sei $x,y\in D$, \OBDA sei $y>x$ und $\lambda\in\pair{0,1}$, $r\definedas\lambda x + \pair{1-\lambda}\cdot y$.
        Es gilt nach Konstruktion, dass $x < r < y$. Wir untersuchen die Intervalle $D_1\definedas\pair{x,r}$ und $D_{2}\definedas\pair{r,y}$. Nach Satz~\ref{satz:mittelwertsatz} gilt $\exists \xi_1\in D_1,\xi_2\in D_2, \xi_1 < \xi_2$ sodass
        \begin{align*}
            \frac{f(r)-f(x)}{r-x} = f'\of{\xi_1} &\leq f'\of{\xi_2} = \frac{f(y)-f(r)}{y-r}
            \intertext{Beachte $r-x = \lambda x + \pair{1-\lambda}\cdot y - x = \pair{1-\lambda}\pair{y-x}$ und $y-r = \lambda\pair{y-x}$}
            \impl \frac{f(r)-f(x)}{\pair{1-\lambda}\pair{y-x}} &\leq \frac{f(y)-f(r)}{\lambda\pair{y-x}}\\
            \impl f(r) &\leq \pair{1-\lambda}\cdot f(y) + \lambda f(x)\\
            \impl f(\lambda x + \pair{1-\lambda}\cdot y) &\leq \pair{1-\lambda}\cdot f(y) + \lambda f(x)\\
            \impl f &\text{ ist konvex}
            \intertext{\anf{$\impl$}: Wir führen einen Beweis mit Kontraposition. Angenommen es existiert ein $x_0\in D$ so, dass $f''(x_0) < 0$. Wir definieren die Hilfsfunktion}
            \varphi\of{x} &= f(x) + c\cdot\pair{x-x_0}\\
            \impl \varphi'\of{x} &= f'\of{x} + c\\
            \impl \varphi'\of{x_0} &= f'\of{x_0} + c
            \intertext{Wähle $c=-f'\of{x_0}$}
            \impl \varphi'\of{x_0} &= 0
            \intertext{Es gilt $\varphi''\of{x_0} = f''\of{x_0} < 0$}
            \impl \varphi \text{ hat in } &x_0\in D \text{ ein isoliertes Maximum}\\
            \impl \exists t > 0\colon \varphi\of{x_0-t} &< \varphi\of{x_0} \land \varphi\of{x_0+t} < \varphi\of{x_0}\\
            \impl f(x_0) = \varphi\of{x_0} &> \frac{1}{2}\pair{\varphi\of{x_0+t} + \varphi\of{x_0-t}}\\
            \impl f(x_0) &> \frac{1}{2}\pair{f(x_0+t)+ct + f(x_0-t)-ct}\\
            \impl f(x_0) &> \frac{1}{2}\pair{f(x_0+t)+f(x_0-t)}
            \intertext{Wir wählen $y=x_0+t$, $x=x_0-t$, $\lambda = \frac{1}{2}$}
            \impl f(\lambda x + \pair{1-\lambda}\cdot y) &> \frac{1}{2}f(x) + \pair{1-\lambda}f(y)\\
            \impl f &\text{ ist nicht konvex}\qedhere
        \end{align*}
    \end{proof}
\end{satz}

\begin{beispiel}
    Wir betrachten $f: \pair{0,\infty}\fromto\R$, $x\mapsto x^r$ für $r\in\R$
    \begin{align*}
        f'\of{x} &= r\cdot x^{r-1}\\
        f''(x) &= r\cdot\pair{r-1}\cdot x^{r-2}
    \end{align*}
    Für $r\geq 1$ und $r\leq 0$ ist $f$ auf $\pair{0,\infty}$ konvex, für $r\in\pair{0,1}$ konkav.
\end{beispiel}

\begin{beispiel}
    Die Funktion $f: \R\fromto\R$, $x\mapsto e^{\lambda x}$
    \begin{align*}
        f''(x) &= \lambda^2\cdot e^{\lambda x} > 0
    \end{align*}
    ist konvex auf $\R$.
\end{beispiel}

\begin{beispiel}
    Die Funktion $f: \pair{0, \infty}\fromto\R$, $x\mapsto\ln\of{x}$
    \begin{align*}
        f''(x) &= \frac{-1}{x^2} < 0
    \end{align*}
    ist konkav auf $\pair{0, \infty}$.
\end{beispiel}

\begin{beispiel}
    Die Funktion $f: \interv{-\frac{\pi}{2}, \frac{\pi}{2}}$, $x\mapsto \arctan\of{x}$
    \begin{align*}
        f'(x) &= \frac{1}{1+x^2} = \pair{1+x^2}^{-1}\\
        \impl f''(x) &= -\pair{1+x^2}^{-2}\cdot 2x
    \end{align*}
    ist konvex im Intervall $\interv{-\frac{\pi}{2}, 0}$ und konkav in $\interv{0, \frac{\pi}{2}}$.
\end{beispiel}

\begin{lemma} % Lemma 2
    \label{lemma:konvex-verkettung}
    Seien $D_1, D_2\sbset\R$ Intervalle und $f: D_1\fromto D_2$ und $g: D_2\fromto \R$. Dann gilt
    \begin{enumerate}[label=(\roman*)]
        \item Falls $f$ konvex, $g$ konvex und monoton wachsend $\impl g\circ f: D_1\fromto\R$ ist konvex
        \item Falls $f$ konkav, $g$ konvex und monoton fallend $\impl g \circ f: D_1\fromto \R$ ist konvex
    \end{enumerate}

    \begin{proof}[Beweis (i)]
        Es seien $x,y\in D_1$ mit $x < y$ und $\lambda\in\pair{0,1}$ dann ist
        \begin{align*}
            \pair{g\circ f}\of{\lambda x + \pair{1-\lambda}\cdot y} &= g\of{f\of{\lambda x + \pair{1-\lambda}\cdot y}}
            \intertext{Wegen der Konvexität von $f$ und der Monotonie von $g$ gilt}
            \impl \pair{g\circ f}\of{\lambda x + \pair{1-\lambda}\cdot y} &\leq g\of{\lambda f\of{x} + \pair{1-\lambda}\cdot f\of{y}}\\
            \impl \pair{g\circ f}\of{\lambda x + \pair{1-\lambda}\cdot y} &\leq \lambda g\of{f\of{x}} + \pair{1-\lambda}\cdot g\of{f\of{y}}\\
            \impl \pair{g\circ f}\of{\lambda x + \pair{1-\lambda}\cdot y} &\leq \lambda\pair{g\circ f}\of{x} + \pair{1-\lambda}\cdot\pair{g\circ f}\of{y}\qedhere
        \end{align*}
    \end{proof}
    \noindent Der Beweis von (ii) funktioniert analog.
\end{lemma}

\begin{lemma} % Lemma 3
    \label{lemma:konvex-umkehrabbildung}
    Es seien $D, B\sbset\R$ Intervalle und $f: D\fromto B$ bijektiv mit $f^{-1}: B\fromto D$ Umkehrabbildung. Dann gilt
    \begin{enumerate}[label=(\roman*)]
        \item $f$ ist monoton wachsend und konvex $\equivalent f^{-1}$ ist monoton wachsend und konkav
        \item $f$ ist monoton fallend und konvex $\equivalent f^{-1}$ ist monoton fallend und konvex
        \item $f$ ist monoton fallend und konkav $\equivalent f^{-1}$ ist monoton fallend und konkav
    \end{enumerate}
\end{lemma}

\begin{uebung}
    Beweisen Sie Lemma~\ref{lemma:konvex-umkehrabbildung}.
\end{uebung}

\newpage

\subsection{Ungleichungen von Jensen und Hölder}
\begin{satz}[Ungleichung von Jensen]
    \label{satz:ungleichung-jensen}
    Sei $f: D\fromto\R$ konvex (konkav), sowie $x_1,\dots, x_n\in D$ und $\lambda_1, \dots, \lambda_n\in\pair{0,1}$ mit $ \sum_{}^{} \lambda_i = 1$. Dann ist
    \begin{align*}
        f\of{\sum_{i=1}^{n} \lambda_i x_i}\underset{(\geq)}{\leq}~ \sum_{i=1}^{n} \lambda_i f\of{x_i}
    \end{align*}

    \begin{proof}
        Wir nutzen vollständige Induktion.\\
        \begin{induktionsanfang}
            $n=2$ gilt wegen Definition der Konvexität.
        \end{induktionsanfang}
        \begin{induktionsvoraussetzung}
            Es gelte die Behauptung für ein festes, aber beliebiges $n\in\N$ mit $n\geq 2$.
        \end{induktionsvoraussetzung}
        \begin{induktionsschritt}
            $n\fromto n+1$
            \begin{align*}
                f\of{\sum_{i=1}^{n+1} \lambda_i x_i} &= f\of{\underbrace{\pair{1-\lambda_{n+1}}}_{1-\theta}\cdot \underbrace{\sum_{i=1}^{n} \frac{\lambda_i}{1-\lambda_{n+1}}\cdot x_i}_{x} + \underbrace{\lambda_{n+1}}_{\theta}\cdot \underbrace{x_{n+1}}_{y}}
                \intertext{Nach der Konvexität von $f$ gilt damit}
                &\leq \pair{1-\lambda_{n+1}}\cdot f\of{\sum_{i=1}^{n} \frac{\lambda_i}{1-\lambda_{n+1}}\cdot x_i} + \lambda_{n+1} \cdot f\of{x_{n+1}}\\
                \annot[{&}]{\leq}{IV} \pair{1-\lambda_{n+1}}\cdot \pair{\sum_{i=1}^{n} \frac{\lambda_i}{1-\lambda_{n+1}}\cdot f\of{x_i}} + \lambda_{n+1}\cdot f(x_{n+1})\\
                &= \sum_{i=1}^{n+1} \lambda_i f\of{x_i}
            \end{align*}
            Nach Induktion folgt die Behauptung.\qedhere
        \end{induktionsschritt}
    \end{proof}
\end{satz}


\begin{korollar}[Ungleichung vom arithmetisch-geometrischen Mittel] % Korollar 1
    Es seien $x_1, \dots, x_n\in \R$. Dann gilt
    \begin{align*}
        \sqrt[n]{\prod_{i=1}^{n} x_i} &\leq \frac{1}{n}\cdot \sum_{i=1}^{n} x_i
    \end{align*}
    \begin{proof}
        $f: \pair{0,\infty}$, $x\mapsto \ln\of{x}$ ist konkav.
        \begin{align*}
            \annot{\impl}{\ref{satz:ungleichung-jensen}} \ln\of{ \sum_{i=1}^{n} \lambda_i x_i} &\geq \sum_{i=1}^{n} \lambda_i \ln\of{x_i}\\
            \impl \sum_{i=1}^{n} \lambda_i x_i &\geq \exp\of{\sum_{i=1}^{n} \lambda_i\cdot \ln\of{x_i}}\\
            &= \prod_{i=1}^{n}  e^{\ln\of{x_i^{\lambda_i}}}\\
            &= \prod_{i=1}^{n}  x_i^{\lambda_i}
            \intertext{Wähle $\lambda_i = \frac{1}{n}$}
            \impl \frac{1}{n}\cdot\sum_{i=1}^{n} x_i &\geq \sqrt[n]{\prod_{i=1}^{n} x_i}\qedhere
        \end{align*}
    \end{proof}
\end{korollar}

\begin{korollar}[Ungleichung von Hölder\footnote{Otto Hölder (1859 - 1937), deutscher Mathematiker, der durch seine Arbeit zur Hölder-Ungleichung, Hölder-Stetigkeit und Kompositionsreihen bekannt ist und der Schwiegervater der Enkelin Marius Sophus Lies.}] % Korollar 2
    \label{korollar:hoelder}
    Seien $p,q\in\pair{1,\infty}$ mit $\frac{1}{p} + \frac{1}{q} = 1$, $x = \pair{x_1, \dots, x_n}\in\C^n$, $y = \pair{y_1, \dots, y_n}\in\C^n$. Wir definieren die $p$-Norm
    \begin{align*}
        \norm{x}_p &\definedas \pair{\sum_{i=1}^{n} \abs{x_i}^p}^{\frac{1}{p}}
        \intertext{Dann gilt}
        \sum_{i=1}^{n} \abs{x_i y_i} &\leq \norm{x}_p \cdot \norm{y}_q
    \end{align*}
    \begin{proof}
        Da $\frac{1}{p} + \frac{1}{q} = 1$ und $\ln$ konkav gilt für beliebige $\zeta, \eta\in\R_+$
        \begin{align*}
            \ln\of{\frac{1}{p} \cdot \zeta + \frac{1}{q} \cdot \eta} &\geq \frac{1}{p}\ln\of{\zeta} + \frac{1}{q}\ln\of{\eta}\\
            \impl \frac{\zeta}{p} + \frac{\eta}{q} &\geq \zeta^{\frac{1}{p}} \eta^{\frac{1}{q}}
            \intertext{Wir definieren}
            \zeta_k \definedas \frac{\abs{x_k}^p}{\norm{x}_p^p}\quad&\quad\eta_n \definedas \frac{\abs{y_n}^q}{\norm{y}_q^q}\\
            \impl \sum_{k=1}^{n} \frac{\abs{x_k}\abs{y_k}}{\norm{x}_p\norm{y}_q} &\leq \sum_{k=1}^{n} \frac{1}{p}\cdot\frac{\abs{x_k}^p}{\norm{x}_p^p} + \frac{1}{q}\cdot\frac{\abs{y_k}^q}{\norm{y}_q^q} = \frac{1}{p}+\frac{1}{q} = 1\\
            \impl \sum_{k=1}^{n} \abs{x_k}\abs{y_k} &\leq \norm{x}_p\cdot \norm{y}_q\qedhere
        \end{align*}
    \end{proof}
\end{korollar}


    \vfill

    \begin{center}
        \textbf{\LARGE TO BE CONTINUED . . .}
    \end{center}

    \vfill

\end{document}
