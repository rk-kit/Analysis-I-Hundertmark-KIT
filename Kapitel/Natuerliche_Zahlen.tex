\thispagestyle{pagenumberonly}
Frage: Was sind die natürlichen Zahlen? (zum Beispiel $1$, $2\definedas 1+1$, $3\definedas 2+1$, $4\definedas 3+1$, usw.)

\subsection{Induktive Mengen}
\begin{definition}[Induktive Menge]
    Eine Teilmenge $A\subseteq \realnumbers$ heißt induktiv, falls
    \begin{enumerate}
        \item $1\in A$ und
        \item Ist $x\in A$ so ist auch $x+1\in A$
    \end{enumerate}
\end{definition}

\begin{beispiel}
    \begin{align*}
        \linterv{0, \infty} &= \set{x|~x\geq 0}\text{ und}\\
        \linterv{1, \infty} &= \set{x|~x\geq 1} \text{ sind induktiv}
    \end{align*}
\end{beispiel}

\begin{beobachtung}[Schnittmengen von induktiven Mengen]
    Ist $J\neq\emptyset$ Indexmenge und $A_j$ induktive Teilmenge von $\realnumbers$ für jedes $j\in J$.
    Dann folgt daraus: $A\definedas \bigcap_{j\in J}A_j$ ist induktiv.\\
    Anders formuliert: Beliebige Schnittmengen von induktiven Mengen sind induktiv.
    \begin{proof}
        \begin{align*}
            x\in A &\equivalent \forall j\in J\colon x \in A_j\\
            &\impl \forall j\in J\colon x+1 \in A_j\\
            &\impl x+1 \in \pair{\bigcap_{j\in J}A_j}=A\qedhere
        \end{align*}
    \end{proof}
\end{beobachtung}

\begin{definition}[Definition von $\naturalnumbers$]
    Sei $\bar{f}\definedas\set{A\subseteq \realnumbers|~A\text{ ist induktiv}}$.
    \begin{align*}
        \naturalnumbers &\definedas \text{kleinste induktive Teilmenge von }\realnumbers\\
        &\definedas \bigcap_{A\in \bar{f}}A
    \end{align*}
    d. h. $\naturalnumbers\subseteq A$, falls $A$ induktiv ist.
\end{definition}

\begin{satz}[Induktionsprinzip]
    \label{satz:induktionsprinzip}
    Ist $M\subseteq \naturalnumbers$ induktiv $\impl M = \naturalnumbers$
    \begin{proof}
        Nach Voraussetzung ist $M\subseteq \naturalnumbers$ und aus der Definition von $\naturalnumbers$ als Schnitt aller induktiven Teilmengen von $\realnumbers$ ist auch $\naturalnumbers\subseteq M \impl \naturalnumbers = M$.
    \end{proof}
\end{satz}

\newpage

%%%%%%%%%%%%%%%%%%%%%%%%
% 9. November 2023
%%%%%%%%%%%%%%%%%%%%%%%%

\subsection{Vollständige Induktion}

\begin{satz}[Induktionsbeweis]
    \marginnote{[9. Nov]}
    Für jedes $n\in\naturalnumbers$ sei eine Aussage $B_n$ gegeben derart, dass folgendes gilt:
    \begin{enumerate}
        \item $B_1$ ist wahr
        \item Aus der Annahme, dass $B_n$ ($n\in\naturalnumbers$) wahr ist folgt, dass $B_{n+1}$ wahr ist.
    \end{enumerate}
    Dann ist $B_n$ wahr $\forall n\in\naturalnumbers$.
    \begin{proof}
        Definiere: $M\definedas\set{n\in\naturalnumbers|~B_n\text{ ist wahr}} \subseteq \naturalnumbers$.\\
        Zu zeigen: $M=\naturalnumbers$. Also reicht es nach Satz~\ref{satz:induktionsprinzip} zu zeigen, dass $M$ induktiv ist.
        \begin{enumerate}
            \item $1\in\naturalnumbers$, da $B_1$ wahr ist.
            \item Ist $n\in M$ dann ist $B_n$ wahr
            \begin{align*}
                &\impl B_{n+1} \text{ ist wahr }\\
                &\impl n+1\in M\\
                &\impl M \text{ ist induktiv } \\
                &\impl M = \naturalnumbers\qedhere
            \end{align*}
        \end{enumerate}
    \end{proof}
\end{satz}

\begin{bemerkung}[Starke Induktion]
    Die starke Induktion ist eine Variante der vollständigen Induktion. Bei dieser beweist man folgendes:
    \begin{enumerate}
        \item $B_1$ ist wahr.
        \item für alle $n\in\naturalnumbers$ gilt $B_1, B_2, B_3, \dots, B_n$ ist wahr $\impl$ $B_{n+1}$ ist wahr.
    \end{enumerate}
\end{bemerkung}

\begin{beispiel}
    Zu zeigen ist: $B_n: n < 2^{n}$
    \begin{proof}
        ~\\
        \begin{induktionsanfang}
            $B_1$ ist wahr, da $1<2^{1}=2$
        \end{induktionsanfang}
        \begin{induktionsschritt}
            Induktionsannahme $B_n$ ist wahr für ein $n=k$, d.h. $k<2^k$. Also zu zeigen: $k+1<2^{k+1}$. $2^{k+1} = 2\cdot 2^{k} > 2k \geq k+1$\qedhere
        \end{induktionsschritt}
    \end{proof}
\end{beispiel}

\begin{uebung}
    Zeigen Sie, dass $2k\geq k+1$ für alle $k\in\naturalnumbers$ gilt.
\end{uebung}

\begin{beispiel}[Gaußsche Summenformel]
    Zu zeigen: $1+2+3+\dots+n= \frac{n\cdot(n+1)}{2}~\forall n\in\naturalnumbers$
    \begin{proof}
        $B_n: 1+2+\dots+n = \frac{n\cdot(n+1)}{2}$\\
        \begin{induktionsanfang}
            $B_1: 1=\frac{1\cdot(1+1)}{2}=\frac{2}{2}=1$
        \end{induktionsanfang}
        \begin{induktionsschritt}
            $B_n$ ist wahr für ein $k\in\naturalnumbers$, d.h. $1+2+\dots+k=\frac{k\cdot(k+1)}{2}$
            \begin{align*}
                1+2+\dots+k+(k+1) \annot[{&}]{=}{(IA)} \frac{k\cdot(k+1)}{2}+(k+1)\\
                &= (k+1)\cdot\pair{\frac{k}{2}+1}\\
                &= \frac{(k+1)\cdot(k+2)}{2}\\
            \end{align*}
        \end{induktionsschritt}
        \noindent Also ist $B_n$ wahr für $n=k+1$. Nach dem Prinzip der vollständigen Induktion folgt $\forall n\in\naturalnumbers\colon~B_n$ ist wahr, d.h. $1+2+\dots+n=\frac{n\cdot(n+1)}{2}$.\qedhere
    \end{proof}
\end{beispiel}

\newpage

\begin{satz}[Eigenschaften von $\naturalnumbers$]
    \label{satz:n-eigenschaften}
    \theoremescape
    \begin{enumerate}
        \item $n\geq 1\quad\forall n\in\naturalnumbers$
        \item $n+m\in\naturalnumbers\quad\forall n,m\in\naturalnumbers$
        \item $n\cdot m\in\naturalnumbers\quad\forall n,m\in\naturalnumbers$
        \item Für $n\in\naturalnumbers$ gilt entweder $n=1$ oder $n-1\in\naturalnumbers$
        \item $(n-m) \in \naturalnumbers\quad\forall n,m\in\naturalnumbers$ mit $m<n$
    \end{enumerate}
    \begin{proof}[Beweis (1.)]
        ~\\
        \begin{induktionsanfang}
            $n\geq 1$ gilt für $n=1$
        \end{induktionsanfang}
        \begin{induktionsschritt}
            Nach Anfang: $n\geq 1$ ist wahr für ein $n=k$.\\
            Da $k+1>k\impl k+1>k\geq 1 \impl k+1\geq 1$\qedhere
        \end{induktionsschritt}
    \end{proof}
    \begin{proof}[Beweis (2.)]
        Fixiere $m\in\naturalnumbers$ für $n\in\naturalnumbers$. Behauptung: $B_n: m+n\in \naturalnumbers$\\
        \begin{induktionsanfang}
            $B_1: m+1 \in \naturalnumbers$, da $\naturalnumbers$ induktiv ist.
        \end{induktionsanfang}
        \begin{induktionsschritt}
            Nach Anfang: $B_n$ ist wahr für $n=k$, d.h. $(m+k)\in\naturalnumbers$. Somit für $k+1: m+(k+1) = (m+k)+1\in\naturalnumbers \impl$ $B_n$ ist wahr $\forall n\in\naturalnumbers$\qedhere
        \end{induktionsschritt}
    \end{proof}
    \begin{proof}[Beweis (3.)]
        Fixiere $m\in\naturalnumbers$ für $n\in\naturalnumbers$. Behauptung: $B_n: m\cdot n\in \naturalnumbers$\\
        \begin{induktionsanfang}
            $B_1: m\cdot 1=m \in \naturalnumbers$
        \end{induktionsanfang}
        \begin{induktionsschritt}
            Nach Anfang: $B_n$ ist wahr für $n=k$, d.h. $m\cdot(k+1) = mk + m \in\naturalnumbers$ gilt nach Satz 2.\qedhere
        \end{induktionsschritt}
    \end{proof}
    \begin{proof}[Beweis (4.)]
        $B_n: n= 1 \lor n-1\in\naturalnumbers$\\
        \begin{induktionsanfang}
            $B_1: 1=1$
        \end{induktionsanfang}
        \begin{induktionsschritt}
            Nehmen an, für ein $n=k$ ist $G_k$ wahr. Also ist entweder (a) $k=1$ oder (b) $(k-1)\in\naturalnumbers$.\\
            Für $n=k+1$ gilt dann im Fall (a): $(k+1)-1 = k\in\naturalnumbers$\\
            Im Fall (b): $(k+1)-1 = (k-1)+1$ und es gilt $(k-1)\in\naturalnumbers$. Daraus folgt, dass $(k-1)+1\in\naturalnumbers$, da $\naturalnumbers$ induktiv ist.\qedhere
        \end{induktionsschritt}
    \end{proof}
    \begin{proof}[Beweis (5.)]
        $B_n: n-m\in\naturalnumbers$ für jedes $m,n\in\naturalnumbers$ mit $m<n$\\
        \begin{induktionsanfang}
            $B_1$ leere Behauptung, da kein $m$ existiert, mit $m<1$ (nach (1.)).
        \end{induktionsanfang}
        \begin{induktionsschritt}
            $B_n$ wahr für ein $n=k$. Das heißt $k-1\in\naturalnumbers~\forall m\in\naturalnumbers$ mit $m<k$.\\
            Zu zeigen: $(k+1)-m\in\naturalnumbers~\forall m<k+1$. Ist $m=1\impl m-1 = 0$ und $(k+1)-m = k+1-m=k\in\naturalnumbers$.\\
            Ist $m>1 \annot{\impl}{(4.)} m-1\in\naturalnumbers$. Da $m<k+1 \impl m-1<k \impl (k+1)-m = k-(m-1)\in\naturalnumbers$ (nach Annahme).\qedhere
        \end{induktionsschritt}
    \end{proof}
\end{satz}

\begin{korollar}
    \label{korollar:4.2.7}
    Es gibt kein $n\in\naturalnumbers$ mit $0<n<1$. Ferner gilt $\forall n\in\naturalnumbers$ gibt es keine natürliche Zahl $m\in\naturalnumbers$ mit $n<m<n+1$ oder mit $n-1<m<n$.
\end{korollar}
\begin{uebung}
    Beweisen Sie das vorherige Korollar mit Satz~\ref{satz:n-eigenschaften}.
\end{uebung}

\newpage

\begin{notation}[Zahlenmengen]
    \begin{align*}
        \naturalnumbers_0 &\definedas \naturalnumbers\cup\set{0}\\
        \mathbb{-N} &\definedas \set{-n|~n\in\naturalnumbers}\\
        \mathbb{Z} &\definedas \naturalnumbers\cup\set{0}\cup\mathbb{-N} \tag{Ganze Zahlen}\\
        \mathbb{Q} &\definedas\set{\frac{p}{q}\middle|~p\in\mathbb{Z}, q\in\naturalnumbers} \tag{Rationale Zahlen}\\
        \realnumbers&\exclude\mathbb{Q}\tag{Irrationale Zahlen}
    \end{align*}
\end{notation}
\begin{bemerkung}
    Sei für $n\in\mathbb{Z}~B_n$ eine Aussage und $n_0\in\naturalnumbers$.\\
    Dann gilt $\pair{\forall n\geq n_0\colon B_n} \equivalent$ ($B_n$ ist wahr) $\land$ (Ist $B_n$ wahr für $n\geq n_0$ so ist auch $B_{n+1}$ wahr)
\end{bemerkung}

\begin{satz}
    $n,k\in\mathbb{Z} \impl a +b \in\mathbb{Z}$ und $a\cdot b \in\mathbb{Z}$
    \begin{proof}
        Folgt aus Satz~\ref{satz:n-eigenschaften} mit $-(-a) = a\quad a > 0 \equivalent -a < 0$
    \end{proof}
\end{satz}

\begin{satz}[Satz von Archimedes]
    \label{satz:von-archimedes}
    $\forall x\in\realnumbers~\exists n\in\naturalnumbers\colon x < n$.
    (Das heißt $\naturalnumbers$ ist eine nach oben unbeschränkte Teilmenge von $\realnumbers$)
    \begin{proof}
        Angenommen die Aussage ist falsch. Das heißt $\exists x\in\realnumbers$ mit $n\leq x\quad\forall n\in\naturalnumbers$.
        \begin{align*}
            \annot[{&}]{\impl}{\ref{axiom:vollstaendigkeitsaxiom}} a\definedas \sup\naturalnumbers \text{ existiert und } n<a\quad\forall n\in\naturalnumbers
            \intertext{Es gilt $a+1 > a \impl a - 1 < a$, das heißt $a-1$ ist keine obere Schranke für $\naturalnumbers$}
            &\impl \exists n\in\naturalnumbers\colon a-1<n\\
            &\impl \exists n\in\naturalnumbers\colon a < n + 1 \in\naturalnumbers
        \end{align*}
        Widerspruch zu $a$ ist obere Schranke für $\naturalnumbers$.
    \end{proof}
\end{satz}

\begin{korollar}
    $\forall x\in\realnumbers~\exists n\in\naturalnumbers\colon -n<x$
    \begin{proof}
        Wende vorherigen Satz auf $-x$ an. $\impl \exists n\in\naturalnumbers\colon -x < n \impl x > -n$
    \end{proof}
\end{korollar}

\begin{satz}[Wohlordnungsprinzip für $\naturalnumbers$]
    \label{satz:wohlordnungsprinzip}
    Jede nichtleere Menge natürlicher Zahlen hat ein kleinstes Element.
    \begin{proof}
        Sei $M\subseteq \naturalnumbers$, $M\neq\emptyset$. Es gilt
        \begin{align*}
            \inf \naturalnumbers = 1 &\impl a = \inf M \geq 1 > -\infty
            \intertext{Zu zeigen: $a\in M$. Annahme: $a\not\in M$}
            &\impl a < m\quad\forall m\in\naturalnumbers
            \intertext{Satz~\ref{satz:inf} besagt, dass $\forall \varepsilon > 0~\exists m\in M\colon m<a+\varepsilon$. Für $\varepsilon=1$ gilt damit}
            &\impl \exists m\in M\colon m < a + 1
            \intertext{Wir wählen $\varepsilon = a - m$}
            &\impl \exists m'\in M\colon m'<a+\varepsilon = m
            \intertext{Das heißt wir haben $m',m\in M$ mit $a<m'<m<a+1$}
            &\impl 0 < m-m' < 1\\
            \text{ aber } (\ref{satz:n-eigenschaften}~(5.)) &\impl m-m'\in\naturalnumbers
        \end{align*}
        Widerspruch zu Korollar~\ref{korollar:4.2.7}.
    \end{proof}
\end{satz}

\newpage