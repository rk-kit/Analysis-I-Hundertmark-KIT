\thispagestyle{pagenumberonly}

\subsection{Cauchyprodukt}
Frage: Gegeben Reihen $\sum_{n=0}^{\infty} a_n$, $\sum_{n=0}^{\infty} b_n$, beide konvergent. Wie kann man das Produkt
\begin{align*}
    \pair{\sum_{n=0}^{\infty} a_n}\cdot\pair{\sum_{n=0}^{\infty} b_n}
\end{align*}
geschickt berechnen?
\begin{alignat*}{2}
    u &= \sum_{n=0}^{\infty} a_n = \lim_{n\fromto\infty} s_n\quad &s_n \definedas \sum_{l=0}^{n} a_l\\
    v &= \sum_{n=0}^{\infty} b_n = \lim_{n\fromto\infty} t_n \quad &t_n \definedas \sum_{j=0}^{n} b_j\\
    \impl u\cdot v &= \lim_{n\fromto\infty} s_n \cdot \lim_{n\fromto\infty} t_n = \lim_{n\fromto\infty} \pair{s_n\cdot t_n}\\[10pt]
    s_n \cdot t_n &= \sum_{l=0}^{n} a_l \cdot \sum_{j=0}^{n} b_j = \sum_{l=0}^{n} \sum_{j=0}^{n} a_l\cdot b_j
\end{alignat*}
Fakt: Indexmenge der Produkte $a_l b_j$ ist $\N_0 \times \N_0$.
\begin{align*}
    \set{(l,j):~l,j\in\N_0} &= \N_0\times\N_0
\end{align*}
Es gibt (viele) Bijektionen $\sigma: \N_0 \fromto \N_0\times\N_0$ zum Beispiel durch Schrägabzählen.\\
Frage: Ist $\sigma: \N_0 \fromto \N_0\times\N_0$ eine beliebige Bijektion, gilt dann
\begin{align*}
    \pair{\sum_{n=0}^{\infty} a_n}\cdot\pair{\sum_{n=0}^{\infty} b_n} &= \sum_{n=0}^{\infty} a_{\sigma_1(n)}\cdot b_{\sigma_2(n)}\\
    \sigma(n) &= \pair{\sigma_1(n), \sigma_2(n)}
\end{align*}
Antwort: Im Allgemeinen nein, aber okay, wenn $\sum_{n}^{} a_n$, $\sum_{n}^{} b_n$ absolut konvergent sind.

\begin{satz} % Satz 1
    \label{satz:cauchyprodukt}
    Seien $\sum_{n=0}^{\infty} a_n$, $\sum_{n=0}^{\infty} b_n$ absolut konvergente Reihen. Dann gilt für jede Bijektion $\sigma: \N_0 \fromto \N_0\times\N_0$
    \begin{align*}
        \pair{\sum_{n=0}^{\infty} a_n} \cdot \sum_{n=0}^{\infty} b_n &= \sum_{n=0}^{\infty} a_{\sigma_1(n)} \cdot b_{\sigma_2(n)}
        \intertext{Das heißt mit $\sigma(n)=\pair{\sigma_1(n), \sigma_2(n)}$}
        c_n &\definedas a_{\sigma_1(n)}\cdot b_{\sigma_2(n)}\\
        u\cdot v &= \sum_{n=0}^{\infty} c_n
        \intertext{mit der Reihe $\sum_{n=0}^{\infty} c_n$ auch absolut konvergent. Ferner gilt}
        u\cdot v = \underbrace{\sum_{l=0}^{\infty} \pair{\sum_{j=0}^{\infty} a_l\cdot b_j}}_{\text{Vertikal abzählen}} &= \underbrace{\sum_{j=0}^{\infty} \pair{\sum_{l=0}^{\infty} a_l\cdot b_j}}_{\text{Horizontal abzählen}} = \underbrace{\sum_{k=0}^{\infty} \pair{\sum_{\substack{0\leq j\\l+j=k}}^{} a_l\cdot b_j}}_{\text{Schräg abzählen}}
    \end{align*}
    \begin{align*}
        M_L &\definedas \max_{0\leq k\leq L}\pair{\max\pair{\sigma_1(k), \sigma_2(k)}}\\
        \sum_{k=0}^{L} \abs{a_{\sigma_1(k)}\cdot b_{\sigma_2(k)}} &\leq \sum_{\substack{0\leq l\leq M_L\\0\leq j\leq M_L}}^{} \abs{a_l\cdot b_j} = \pair{\sum_{l=0}^{M_L} \abs{a_l}} \cdot \pair{\sum_{j=0}^{M_L} \abs{b_j}}\\
        &\leq \pair{\sum_{l=0}^{\infty} \abs{a_l}} \cdot \pair{\sum_{j=0}^{\infty} \abs{b_j}} < \infty\\
    \end{align*}
    \marginnote{[11. Jan]}
    Insbesondere gilt
    \begin{align*}
        \pair{\sum_{j=0}^{\infty} a_j}\cdot\pair{\sum_{k=0}^{\infty} b_k} &= \sum_{n=0}^{\infty} \pair{\sum_{\substack{j,k\geq 0\\j+k=n}}^{} a_j\cdot b_k}\\
        &= \sum_{n=0}^{\infty} \pair{a_n b_0 + a_{n-1} b_1 + \dots + a_0 b_n}\tag{Cauchy-Produkt}
    \end{align*}

    %%%%%%%%%%%%%%%%%%%%%%%%
    % 11. Januar 2024
    %%%%%%%%%%%%%%%%%%%%%%%%

    \begin{proof}
        Schritt 1:
        \begin{align*}
            k' \definedas \sum_{j=0}^{\infty} \abs{a_j} &< \infty\\
            k'' \definedas \sum_{k=0}^{\infty} \abs{b_k} &< \infty\\
            \pair{\sum_{j=0}^{n} \abs{a_j}}\cdot\pair{\sum_{k=0}^{n} \abs{b_k}} &= \sum_{j=0}^{n} \sum_{k=0}^{n} \abs{a_j}\abs{b_k} \leq k'\cdot k''\quad\forall n\in\N_0
            \intertext{Sei}
            \sigma: \N_0&\fromto\N\times\N
            \intertext{eine beliebige Bijektion. Behauptung 1: $ \sum_{n=0}^{\infty} c_n$ ist absolut konvergent, wobei}
            c_n &\definedas a_{\sigma_1(n)}b_{\sigma_2(n)} = a_j b_k\tag{$j=\sigma_1(n)$, $k=\sigma_2(n)$}
            \intertext{Sei $L\in\N_0$}
            M_L^1 &\definedas \max_{0\leq n\leq L}\pair{\sigma_1(n)}\quad M_L^2 \definedas \max_{0\leq n\leq L}\pair{\sigma_2(n)}\\
            M_L &\definedas \max\pair{M_L^1, M_L^2}\\[10pt]
            \impl \abs{c_n} &\leq \abs{a_j}\abs{b_n} \text{ für } 0\leq j \leq M_L^1, 0\leq n \leq M_L^2\\
            \sum_{n=0}^{L} \abs{c_n} &\leq \sum_{j=0}^{M_L} \sum_{k=0}^{M_L} \abs{a_j}\abs{b_k}\\
            &= \pair{\sum_{j=0}^{M_L} \abs{a_j}}\cdot\pair{\sum_{k=0}^{M_L} \abs{b_k}} \leq k' \cdot k'' < \infty\quad\forall L\in\N_0\\
            \impl \sum_{n=0}^{\infty} \abs{c_n} &= \sup_{L\in\N_0} \sum_{n=0}^{L} \abs{c_n} \leq k'\cdot k''<\infty
            \intertext{Schritt 2:}
            s&\definedas \sum_{n=0}^{\infty} c_n \in\R
            \intertext{unabhängig von der Reihenfolge, in der man die $c_n$ aufsummiert, wegen absoluter und damit unbedingter Konvergenz. Das heißt für jede Bijektion $\kappa: \N_0\fromto\N_0\times\N_0$ ist}
            s &= \sum_{n=0}^{\infty} a_{\kappa_1(n)}b_{\kappa_2(n)}\tag{$\sigma\circ\kappa^{-1}$ Bijektion}
            \intertext{Schritt 3: Wir zeigen, dass}
            s &= \pair{\sum_{n=0}^{\infty} a_n}\cdot\pair{\sum_{n=0}^{\infty} b_n}
            \intertext{Sei $\sigma:\N_0\fromto\N_0\times\N_0$ Bijektion durch Quadratabzählen. ($c_0=a_{0} b_0$, $c_1 = a_1 b_0$, $c_2=a_1, b_1$, $c_3=a_0 b_2$)}
            \text{Partialsumme } \sum_{k=0}^{n} c_k &= \sum_{k=0}^{n} a_{\sigma_1(k)} b_{\sigma_2(k)} \text{ mit $\sigma$ Quadratabzählen }
            \intertext{Wir betrachten die Folge $\displaystyle\sum_{j=0}^{L} \sum_{k=0}^{L} a_j b_k$. Da $\displaystyle\sum_{j=0}^{L} \sum_{k=0}^{L} a_j b_k$ die Summe der geschlossenen Quadrate ist, ist diese eine Teilfolge von $\sum_{k=0}^{n} c_k$ und beide haben den gleichen Grenzwert.}
            \impl s=\sum_{k=0}^{n} c_k &= \lim_{L\fromto\infty} \sum_{j=0}^{L} \sum_{k=0}^{L} a_j b_k = \lim_{L\fromto\infty} \pair{\sum_{j=0}^{L} a_j}\cdot\pair{\sum_{k=0}^{L} b_k}\\
            &= \pair{\sum_{j=0}^{\infty} a_j}\cdot\pair{\sum_{k=0}^{\infty} b_k}
            \intertext{Schritt 4: Cauchy-Produkt}
            \sum_{n=0}^{L} \sum_{\substack{0\leq j,k\\ j+k=n}}^{} a_j b_k &= \sum_{n=0}^{L} \pair{\sum_{j=0}^{L} a_j b_{L-j}}
            \intertext{ist Teilfolge der Folge $\sum_{n}^{} c_n$ mittels Schrägabzählen}
            \impl \lim_{L\fromto\infty} \sum_{n=0}^{L} \sum_{j=0}^{L} a_j b_{L-j} &= \sum_{n=0}^{\infty} c_{\sigma(n)} = s\qedhere
        \end{align*}
    \end{proof}
\end{satz}

\newpage

\par\noindent\rule[0.25\baselineskip]{.37\textwidth}{0.4pt}\hfill Einschub: Abzählungen\hfill\rule[0.25\baselineskip]{.37\textwidth}{0.4pt}

\begin{definition}[Unendliche Mengen]
    Es sei $A_n\definedas\set{1,2,\dots, n}$. Eine Menge $B$ ist unendlich groß, wenn $B\neq\emptyset$ und keine Bijektion $\kappa: A_n \fromto B$ für ein beliebiges $n$ existiert.
\end{definition}

\begin{beispiel}[Vergleich von Kardinalitäten unendlicher Mengen]
    Wir wollen zeigen, dass $\linterv{0,1}$ und $\interv{0,1}$ gleich groß sind. Wir können alle Zahlen auf sich selber abbilden außer der 1. Wir versuchen $1\mapsto\frac{1}{2}$, $\frac{1}{2}\mapsto\frac{1}{3}$, $\frac{1}{3}\mapsto\frac{1}{4}$, \dots.\\
    Damit können wir alle rationalen Zahlen, die sich als Bruch mit 1 im Zähler darstellen lassen, verschieben. Wir definieren:
    \begin{align*}
        \sigma: \interv{0,1}&\fromto\linterv{0,1}\\
        x&\mapsto
        \begin{cases}
            \frac{1}{n+1},\quad &x=\frac{1}{n} \text{ mit } n\in\N\\
            x,\quad &x\in\interv{0,1}\exclude(\bigcup_{n\in\N} \frac{1}{n})
        \end{cases}
    \end{align*}
\end{beispiel}

\begin{bemerkung}[Beispiel für eine Abzählung von $\N\times\N$]
    \label{bem:abzaehlen-nxn}
    Wir wollen eine bijektive Abbildung $\sigma: \N\fromto\N\times\N$ konstruieren.\\
    Level $l\in\N: A_l = \set{\pair{j,k}: j+k = l+1,~j,k\in\N}$ (schrägen Diagonalen).\\
    Anzahl Punkte in $\N\times\N$ auf Level $l\leq k$ mit
    \begin{align*}
        \sum_{l=1}^{k} l &= \frac{k(k+1)}{2}
        \intertext{Schreibe}
        n &= \frac{k(k+1)}{2} + r\quad r\in\set{0,1,2,\dots, k}, k\in\N
        \intertext{Das ist eine eindeutige Zerlegung von $\N$. Definiere}
        \sigma\of{n} &= \pair{\sigma_1(n), \sigma_2(n)}\\
        &\definedas \pair{k-r, r}\\
        \sigma_1(n) &= k-r\\
        \sigma_2(n) &= r
    \end{align*}
\end{bemerkung}

\begin{uebung}
    Weisen Sie die Bijektivität der definierten Funktion $\sigma$ aus Bemerkung~\ref{bem:abzaehlen-nxn} nach.
\end{uebung}

\par\noindent\rule{\textwidth}{0.4pt}

\newpage

\subsection{Exponentialfunktionen}\label{subsec:exp}

%%%%%%%%%%%%%%%%%%%%%%%%
% 16. Januar 2024
%%%%%%%%%%%%%%%%%%%%%%%%

\begin{definition}[Exponentialfunktion]
    \marginnote{[16. Jan]}
    Es sei $z\in\C$. Dann gilt
    \begin{align*}
        e^z = \exp(z)\definedas \sum_{n=0}^{\infty} \frac{z^n}{n!}
    \end{align*}
    Außerdem ist $z^0=1$.
\end{definition}

\begin{satz}[Eigenschaften der Exponentialfunktion]
    \theoremescape
    \begin{enumerate}[label=(\alph*)]
        \item Für alle $z\in\C$ konvergiert die obige Reihe absolut. (Wohldefiniertheit der $\exp$-Funktion)
        \item Es gilt $\exp(z_1)\cdot\exp(z_2) = \exp(z_1+z_2)$. Insbesondere ist $\exp(z)\neq 0~\forall z\in\C$ und $\exp(z)^{-1} = \exp(-z)$.
        \item $\conj{\exp(z)} = \exp(\conj{z})$
        \item $\abs{\exp(z)} = \exp(\Re(z))$
        \item $e^x = \exp(x) > 0\quad\forall x\in\R$
    \end{enumerate}

    \begin{proof}[Beweis (a)]
        Wir zeigen, dass die Reihe absolut konvergiert.
        \begin{align*}
            a_n &= \frac{1}{n!}\cdot z^n
            \intertext{Nach dem Quotientenkriterium (\ref{satz:quotientenkriterium})}
            \abs{\frac{a_{n+1}}{a_n}} &= \frac{z}{n+1}\fromto 0 \text{ für } n\fromto\infty\\
            \impl \sum_{n=0}^{\infty} a_n &= \sum_{n=0}^{\infty} \frac{z^n}{n!}\text{ konvergiert absolut}\qedhere
        \end{align*}
    \end{proof}
    \begin{proof}[Beweis (b)]
        \begin{align*}
            \exp(z_1) \cdot \exp(z_2) &= \pair{\sum_{j=0}^{\infty} \frac{z^j}{j!}} \cdot \pair{\sum_{k=0}^{\infty} \frac{z^k}{k!}} \annot{=}{\ref{satz:cauchyprodukt}} \sum_{n=0}^{\infty} \sum_{\nu=0}^{n} a_{\nu} b_{n-\nu}\\
            &= \sum_{n=0}^{\infty} \sum_{\nu=0}^{n} \frac{(z_1)^{\nu}}{\nu!}\cdot \frac{(z_2)^{n-\nu}}{(n-\nu)!}\\
            &= \sum_{n=0}^{\infty} \frac{1}{n!}\cdot \underbrace{\sum_{\nu=0}^{n} \frac{n!}{\nu!\cdot(n-\nu)!}\cdot (z_1)^{\nu}\cdot (z_2)^{n-\nu}}_{\text{Binomischer Lehrsatz}}\\
            &= \sum_{n=0}^{\infty} \frac{1}{n!}\cdot(z_1+z_2)^n = \exp(z_1 + z_2)\qedhere
            \intertext{Insbesondere}
            \exp(z)\cdot \exp(-z) &= \exp(z-z) = e^0 = 1\\
            \impl \Bigg\{
            \begin{split}
                \exp(z), \exp(-z) &\neq 0 \quad \forall z\in\C\\
                \exp(z)^{-1} &= \exp(-z)
            \end{split}
        \end{align*}
    \end{proof}
    \begin{proof}[Beweis (c)]
        \begin{align*}
            \conj{\exp(z)} &= \conj{\sum_{k=0}^{\infty} \frac{z^k}{k!}} = \sum_{k=0}^{\infty} \conj{\frac{z^k}{k!}} = \sum_{k=0}^{\infty} \frac{\conj{z^k}}{k!} = \sum_{k=0}^{\infty} \frac{\pair{\conj{z}}^k}{k!} = \exp(\conj{z})\qedhere
        \end{align*}
    \end{proof}
    \begin{proof}[Beweis (d)]
        \begin{align*}
            \abs{\exp(z)}^2 &= \conj{\exp(z)}\cdot \exp(z) = \exp(\conj{z})\cdot \exp(z)\\
            &= \exp(\conj{z} + z) = \exp(2\cdot \Re(z))\\
            &=\exp(\Re(z)+ \Re(z)) = \pair{\exp(\Re(z))}^2\\[10pt]
            \impl \abs{\exp(z)} &= \abs{\exp(\Re(z))} = \exp(\Re(z))\qedhere
        \end{align*}
    \end{proof}
    \begin{proof}[Beweis (e)]
        \begin{align*}
            \text{Ist } x \geq 0 &\impl \exp(x) = \sum_{n=0}^{\infty} \frac{x^n}{n!}= 1 + \sum_{n=1}^{\infty} \frac{x^n}{n!}\geq 1\\
            \text{Ist } x < 0 &\impl \exp(x) = \frac{1}{\exp(-x)} > 0\qedhere
        \end{align*}
    \end{proof}
\end{satz}

\begin{satz}[Definition von Sinus und Kosinus über Expontentialfunktionen]
    Für $\alpha\in\R$ ist $\abs{\exp(i\alpha)}=1$. Wir setzen
    \begin{align*}
        \cos(\alpha) \definedas \Re (e^{i\alpha}) &= \frac{1}{2} \cdot\pair{e^{i\alpha} + e^{-i\alpha}}\\
        \sin(\alpha) \definedas \Im (e^{i\alpha}) &= \frac{1}{2i}\cdot\pair{e^{i\alpha}-e^{-i\alpha}}
        \intertext{Dann haben wir}
        -1 \leq \cos \alpha &\leq 1\\
        -1 \leq \sin \alpha &\leq 1
        \intertext{Außerdem gilt $\forall \alpha\in\R$}
        \cos(\alpha)^2 + \sin(\alpha)^2 &= 1\\
        \cos(\alpha) + i\cdot\sin(\alpha) &= e^{i\alpha} \tag{Eulersche Gleichung}
    \end{align*}

    \begin{proof}
        Es sei $\alpha\in\R$.
        \begin{align*}
            \abs{\exp(i\alpha)}^2 &= \conj{\exp(i\alpha)}\cdot \exp(i\alpha) = \exp(-i\alpha)\cdot \exp(i\alpha) = \exp(0) = 1\\[10pt]
            \Re(e^{i\alpha}) &= \frac{1}{2}\cdot\pair{e^{i\alpha} + \conj{e^{i\alpha}}} = \frac{1}{2}\cdot\pair{e^{i\alpha} + e^{-i\alpha}} \definedasbackwards \cos \alpha\\
            \Im(e^{i\alpha}) &= \frac{1}{2i}\cdot\pair{e^{i\alpha} - \conj{e^{i\alpha}}}= \frac{1}{2i}\cdot\pair{e^{i\alpha} - e^{-i\alpha}}\definedasbackwards \sin \alpha\\[10pt]
            \impl \abs{\exp(i\alpha)}^2 &= \pair{\Re(\exp(i\alpha))}^2 + \pair{\Im(\exp(i\alpha))}^2\\
            &= \pair{\cos (\alpha)}^2 + \pair{\sin (\alpha)}^2\\[10pt]
            \exp(i\alpha) &= \Re (\exp(i\alpha)) + i\cdot \Im (\exp(i\alpha)) = \cos \alpha + i \cdot \sin \alpha\qedhere
        \end{align*}
    \end{proof}
\end{satz}

\newpage