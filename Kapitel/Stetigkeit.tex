Um zu kennzeichnen, dass Aussagen sowohl auf $\R$ als auch auf $\C$ valide sind, stehe ab sofort $\K$ stellvertretend für $\R$ oder $\C$ (Körper).\\
Es sei $D\subseteq \R$ (oder $D\subseteq \C$, wenn wir eine komplexe Funktion betrachten) und $D\neq\emptyset$.

\subsection{Das $\varepsilon$-$\delta$-Kriterium}
\thispagestyle{pagenumberonly}

\begin{definition}[$\varepsilon-\delta$ Definition von Stetigkeit]
    Sei $f: D \fromto \K$. Die Funktion $f$ heißt stetig in $x_0$, falls
    \begin{align*}
        \forall \varepsilon > 0~\exists \delta > 0\colon \abs{f(x)-f(x_0)} < \varepsilon \text{ für alle } x\in D \text{ mit } \abs{x-x_0} < \delta
    \end{align*}
    $f$ ist stetig in $D$, falls es in jedem Punkt $x_0\in D$ stetig ist.
\end{definition}

\begin{bemerkung}
    \theoremescape
    \begin{enumerate}[label=\arabic*)]
        \item Wir können Stetigkeit auch etwas anschaulicher betrachten: $x$ ist der Input, $f(x)$ der Output. $f$ ist genau dann stetig in $x_0$, wenn der Output $f(x)$ $\varepsilon$-nahe bei $f(x_0)$ ist, sofern Input $x$ $\delta$-nahe bei $x_0$ ist.
        \item Eine Funktion $f: D\fromto\K^d$ ist stetig in $x_0\in D$, falls
        \begin{align*}
            \forall \varepsilon>0~\exists\delta>0\colon \underbrace{\norm{f(x)-f(x_0)}}_{\text{Norm}} &< \varepsilon\qquad \forall x\in D\colon \abs{x-x_0} <\delta
            \intertext{oder $f:D\fromto V$ (dabei sei $V$ Vektorraum mit Norm $\norm{\cdot}$) ist stetig in $x_0$, falls}
            \forall \varepsilon>0~\exists\delta>0\colon \norm{f(x)-f(x_0)} &< \varepsilon\qquad \forall x\in D\colon \abs{x-x_0} <\delta
        \end{align*}
        \item Stetigkeit ist eine lokale Eigenschaft.
    \end{enumerate}
\end{bemerkung}

\begin{beispiel}[Dirichlet-Funktion]
    Die Funktion
    \begin{align*}
        f(x) &\definedas \begin{cases}
                             1\quad &x\in\Q\\
                             0\quad &x\in\R\exclude\Q
        \end{cases}
    \end{align*}
    ist nirgends stetig auf $\R$.
\end{beispiel}
\begin{beispiel}
    Die Funktion
    \begin{align*}
        f(x) &\definedas \begin{cases}
                             x\quad &x\in\Q\\
                             0\quad &x\in\R\exclude\Q
        \end{cases}
    \end{align*}
    ist nur stetig in $x_0=0$.
\end{beispiel}
\begin{beispiel}
    \begin{align*}
        f(x) &\definedas \begin{cases}
                             \sin\pair{\frac{1}{x}}\quad &x\neq 0\\
                             0\quad &x = 0
        \end{cases}
    \end{align*}
    ist nicht stetig in 0.
\end{beispiel}

\newpage

\begin{lemma} % Lemma 2
    \label{lemma:abschaetzung-stetigkeit}
    Ist $f:D\fromto\K$ (oder $\K^d$) stetig in $x_0\in D$. Dann gilt
    \begin{enumerate}[label=\arabic*.]
        \item $\exists \delta > 0$ mit $\abs{f(x)}\leq 1+\abs{f(x_0)}$\quad$\forall x\in D\colon \abs{x-x_0} < \delta$
        \item Ist $f(x_0) \neq 0$ dann existiert ein $\delta>0$ sodass
        \begin{align*}
            \frac{1}{2}\abs{f(x_0)} &\leq \abs{f(x)} \leq \frac{3}{2} \abs{f(x_0)} \text{ für } x\in D \text{ mit } \abs{x-x_0} < \delta
        \end{align*}
    \end{enumerate}

    \begin{proof}[Beweis (1.)]
        Wähle $\varepsilon = 1$
        \begin{align*}
            \impl \exists\delta > 0\colon &\abs{f(x)-f(x_0)} < \varepsilon\tag{$x\in D,~\abs{x-x_0} < \delta$}\\[10pt]
            \impl \abs{f(x)} = &\abs{f(x)-f(x_0)+f(x_0)}\\
            \leq &\underbrace{\abs{f(x)-f(x_0)}}_{< 1} + \abs{f(x_0)}\leq 1 + \abs{f(x_0)}\qedhere
        \end{align*}
    \end{proof}
    \begin{proof}[Beweis (2.)]
        Ist $f(x_0) \neq 0$, wähle $\varepsilon = \frac{1}{2}\cdot\abs{f(x_0)} > 0$, $\delta > 0$.
        \begin{align*}
            \abs{f(x)-f(x_0)} &< \varepsilon = \frac{1}{2} \abs{f(x_0)} \tag{$x\in D, \abs{x-x_0} < \delta$}\\[10pt]
            \impl \abs{f(x)} &= \abs{f(x_0)+f(x)-f(x_0)}\\
            &\leq \abs{f(x_0)} + \abs{f(x) - f(x_0)}\\
            &\leq \abs{f(x_0)}+\frac{1}{2}\abs{f(x_0)} = \frac{3}{2}\abs{f(x_0)}\\[10pt]
            \abs{f(x)} &= \abs{f(x_0)+f(x)-f(x_0)}\\
            &\geq \abs{f(x_0)} - \abs{f(x)-f(x_0)}\\
            &\geq \abs{f(x_0)} - \frac{1}{2} \abs{f(x_0)} = \frac{1}{2}\abs{f(x_0)}\qedhere
        \end{align*}
    \end{proof}
\end{lemma}

\begin{notation}
    Seien
    \begin{align*}
        f&: D_f \fromto \K\\
        g&: D_g \fromto \K
    \end{align*}
    Wir definieren
    \begin{align*}
        D_{f+g} &\definedas D_f \cap D_g\\
        D_{\frac{f}{g}} &\definedas D_{f+g} \exclude\set{x\in D_g: g(x)=0}
    \end{align*}
    und setzen
    \begin{align*}
        f + g: D_{f+g} &\fromto \K,\quad x \mapsto  f(x) + g(x)\\
        f \cdot g: D_{f+g} &\fromto \K,\quad x \mapsto  f(x) \cdot g(x)\\
        \frac{f}{g}: D_{\frac{f}{g}} &\fromto \K,\quad x \mapsto \frac{f(x)}{g(x)}
    \end{align*}
\end{notation}

\begin{bemerkung}
    $f+g$ geht auch für $f: D_f \fromto \K^d$, $g: D_g \fromto \K^d$. $f\cdot g$ und $\frac{f}{g}$ gehen auch für $f: D_f \fromto \K^d$, $g: D_g \fromto \K$.
\end{bemerkung}

\newpage

\begin{satz} % Satz 3
    \label{satz:stetigkeit-arithmetik}
    Seien $f, g: D\fromto \K$ stetig in $x_0\in D$. Dann gilt
    \begin{enumerate}[label=\arabic*.]
        \item $f+g: D\fromto\K,~x\mapsto f(x)+g(x)$ ist stetig in $x_0$.
        \item $f\cdot g: D\fromto\K,~x\mapsto f(x)\cdot g(x)$ ist stetig in $x_0$.
        \item Ist $g(x_0)\neq 0$, dann ist $\frac{f}{g}: D_{\frac{f}{g}}\fromto\K,~x\mapsto \frac{f(x)}{g(x)}$ stetig in $x_0$.
    \end{enumerate}

    \begin{proof}[Beweis (1.)]
        \begin{align*}
            \abs{f(x)+g(x)-(f(x_0)+g(x_0))} &\leq \abs{f(x)-f(x_0)} + \abs{g(x)-g(x_0)}
            \intertext{Nach der Stetigkeit der beiden einzelnen Funktionen gilt}
            \forall\varepsilon>0~\exists \delta_1, \delta_2 > 0\colon \abs{f(x)-f(x_0)} &< \frac{\varepsilon}{2}\tag{$x\in D, \abs{x-x_0} < \delta_1$}\\
            \abs{g(x)-g(x_0)} &< \frac{\varepsilon}{2}\tag{$x\in D, \abs{x-x_0} < \delta_2$}
            \intertext{Ist $\delta\definedas\min \set{\delta_1, \delta_2} > 0$}
            \impl \forall\varepsilon>0\colon \abs{f(x)+g(x)-(f(x_0)+g(x_0))} &< \frac{\varepsilon}{2} + \frac{\varepsilon}{2} = \varepsilon\tag{$x\in D, \abs{x-x_0} < \delta$}
        \end{align*}
    \end{proof}

    \begin{proof}[Beweis (2.)]
        Aus Lemma~\ref{lemma:abschaetzung-stetigkeit} folgt
        \begin{align*}
            \exists \delta_0 > 0\colon \abs{g(x)} &\leq 1 + \abs{g(x_0)} \qquad\forall x\in D\colon \abs{x-x_0} < \delta_0
            \intertext{Mit der Stetigkeit von $f$ und $g$ in $x_0$ gilt}
            \forall\varepsilon >0~\exists \delta_1, \delta_2 > 0\colon \abs{f(x)-f(x_0)} &< \frac{\varepsilon}{2\cdot(1+\abs{g(x_0)})}\tag{$x\in D, \abs{x-x_0} < \delta_1$}\\
            \abs{g(x)-g(x_0)} &< \frac{\varepsilon}{2\cdot(1+\abs{f(x)})}\tag{$x\in D, \abs{x-x_0} < \delta_2$}
            \intertext{Wir definieren $\delta\definedas \min(\delta_0, \delta_1, \delta_2)$}
            \impl \abs{f(x)\cdot g(x)-(f(x_0)\cdot g(x_0))} &\leq \abs{f(x)-f(x_0)}\cdot\abs{g(x)} + \abs{f(x_0)}\cdot\abs{g(x)-g(x_0)}\\
            &\leq \frac{\varepsilon}{2\cdot(1+\abs{g(x_0)})}\cdot \abs{g(x)} + \frac{\varepsilon}{2\cdot(1+\abs{f(x)})}\cdot \abs{f(x_0)}\\
            &\leq \frac{\varepsilon}{2\cdot(1+\abs{g(x_0)})}\cdot \pair{1+\abs{g(x_0)}} + \frac{\varepsilon}{2\cdot(1+\abs{f(x)})}\cdot \pair{1+\abs{f(x)}}\\
            &\leq \frac{\varepsilon}{2} + \frac{\varepsilon}{2} = \varepsilon\tag{$x\in D, \abs{x-x_0} < \delta$}
        \end{align*}
    \end{proof}

    \begin{proof}[Beweis (3.)]
        Es reicht aus, zu zeigen, dass $\frac{1}{g(x)}$ in $x_0$ stetig ist. Dann können wir (2.) anwenden. Wir wählen den Ansatz
        \begin{align*}
            \abs{\frac{1}{g(x)} - \frac{1}{g(x_0)}} &= \frac{\abs{g(x_0)-g(x)}}{\abs{g(x)\cdot g(x_0)}}
        \end{align*}
        und schätzen den Term geschickt ab.
    \end{proof}
    \begin{uebung}
        Beweisen Sie Teil (3.) des vorherigen Satzes mit einer geschickten Abschätzung.
    \end{uebung}
\end{satz}

%%%%%%%%%%%%%%%%%%%%%%%%
% 23. Januar 2023
%%%%%%%%%%%%%%%%%%%%%%%%

\subsection{Stetige Funktionen und Folgenkriterium}

\begin{satz}[Stetigkeit von Polynomen] % Satz 3
    \marginnote{[23. Jan]}
    \label{stetigkeit:polynome}
    Für eine stetige Funktion $f$ und eine Zahl $a$, ist auch $a\cdot f$ stetig. Außerdem ist nach Satz~\ref{satz:stetigkeit-arithmetik} $x^2$ bzw. $z^2$ stetig. Es lässt sich induktiv zeigen, dass $x^n$ bzw. $z^n$ damit auch stetig sein müssen.\\
    Daraus folgt, dass alle Polynome (in $\K$) stetig auf ganz $\K$ sind. Außerdem ist für $P,Q$ Polynome mit $Q(x)\neq 0$ auch die Funktion
    \begin{align*}
        R(x) &= \frac{P(x)}{Q(x)}
    \end{align*}
    stetig auf $\K\exclude\set{x\in\K: Q(x) = 0}$.
\end{satz}

\begin{korollar} % Korollar 4
    Alle Polynome (reell oder komplex) sind stetig und alle rationalen Funktionen $R(x) = \frac{P(x)}{Q(x)}$ sind stetig auf $D_R \definedas \K\exclude\set{x\in\K: Q(x) = 0}$.
\end{korollar}

\begin{uebung}[Stetigkeit von Exponentialfunktionen]
    Es gilt
    \begin{align*}
        e^x-e^0 &= e^x - 1 = \sum_{n=0}^{\infty} \frac{x^n}{n!}-1 = \sum_{n=1}^{\infty} \frac{x^n}{n!}
    \end{align*}
    Weisen Sie basierend auf dieser Gleichung und den Abschätzungen für Potenzreihen die Stetigkeit der Funktion $e^x$ zunächst im Punkt $x_0 = 0$ und dann auf ganz $\R$ nach.
\end{uebung}

\begin{satz}[Folgenkriterium für Stetigkeit] % Satz 5
    \label{satz:stetigkeit-folgenkriterium}
    Sei $f: D\fromto\K$ (oder $\K^d$) mit $D\subseteq\K$. Dann gilt $f$ ist genau dann stetig in $x_0$, wenn für alle Folgen $(x_n)_n \subseteq D$ mit $x_n \fromto x_0$ folgt, dass $f(x_n) \fromto f(x_0)$ für $\ntoinf$.

    \begin{proof}
        \anf{$\impl$}: $f$ ist stetig in $x_0$. Dann gilt
        \begin{align*}
            \forall\varepsilon > 0~\exists\delta>0\colon \abs{f(x)-f(x_0)} &< \varepsilon\tag{$x\in D$, $\abs{x-x_0}<\delta$}
            \intertext{Sei $(x_n)_n\subseteq D$ Folge mit $x_n\fromto x_0$. Das heißt}
            \forall \hat{\varepsilon} > 0~\exists N\in\N\colon \abs{x_n-x_0} &< \hat{\varepsilon}\quad\forall n\geq N\\
            \intertext{Gegeben $\varepsilon$ verwenden wir die Stetigkeits-Eigenschaften}
            \exists \delta>0\colon \abs{f(x)-f(x_0)} &< \varepsilon\quad \forall x\in D\colon \abs{x-x_0} < \delta\\
            \intertext{nehmen $\hat{\varepsilon} = \delta$}
            \impl \exists N\in\N\colon \abs{x_n-x_0} &< \delta\quad\forall n\geq N\\
            \impl \abs{f(x_n)-f(x_0)} &< \varepsilon\quad\forall n\geq N_0
        \end{align*}
        Mit der Definition der Konvergenz folgt dann $f(x_n) \fromto f(x_0)$.\\[10pt]
        \anf{$\Leftarrow$}: Wir verwenden Kontraposition. Wenn $f$ nicht stetig in $x_0$ ist, dann
        \begin{align*}
            \impl \exists \varepsilon > 0~\forall \delta > 0~\exists x\in D\colon& \pair{\abs{x-x_0} < \delta} \land \pair{\abs{f(x)-f(x_0)} \geq \varepsilon > 0}
            \intertext{Wählen $\delta = \frac{1}{n}$, $n\in\N$}
            \impl \fa n\in\N \ex x_n\in D\colon& \abs{x_n-x_0} < \frac{1}{n} \text{ und } \abs{f(x_n)-f(x_0)} \geq \varepsilon
        \end{align*}
        Das heißt wir haben ein $(x_n)_n\subseteq D, x_n \fromto x_0$ und $f(x_n)$ konvergiert nicht gegen $f(x_0)$.
    \end{proof}
\end{satz}

\begin{satz}[Stetigkeit unter Verkettung] % Satz 6
    \label{satz:verkettung-stetigkeit}
    Sei $f: D_f \fromto\R$, $D_g\subseteq\R$, $g: D_g\fromto\R$ und $f(D_f)\subseteq D_g$. Sind $f$ stetig in $x_0$ und $g$ stetig in $y_0\definedas f(x_0)\in D_g$. Dann ist $g\circ f: D_f\fromto\R$ stetig in $x_0$.

    \begin{proof}
        Sei $(x_n)_n\subseteq D_f$ mit $x_n\fromto x_0$. Dann folgt aus Satz~\ref{satz:stetigkeit-folgenkriterium}, dass $y_n \definedas f(x_n) \fromto f(x_0) = y_0$. Wenn wir den Satz noch ein zweites Mal anwenden, muss gelten, dass $g(y_n)\fromto g(y_0)$. Also gilt für alle Folgen $(x_n)_n\subseteq D_f$, dass
        \begin{align*}
        (g\circ f)
            \of{x_n} = g(f(x_n)) = g(y_n) \fromto g(y_0) = g(f(x_0)) = (g\circ f)\of{x_0}
        \end{align*}
        Nach Satz~\ref{satz:stetigkeit-folgenkriterium} ist damit $g\circ f$ stetig in $x_0$.
    \end{proof}
\end{satz}

\begin{uebung}
    Zeigen Sie Satz~\ref{satz:verkettung-stetigkeit} mit dem $\varepsilon$-$\delta$-Kriterium der Stetigkeit.
\end{uebung}

\newpage